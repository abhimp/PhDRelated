\section{\textbf{Background and Related Work}}\label{sec:chap04:related_work}
\indent \ac{4G} LTE smartphones are designed to maintain network connectivity using a \ac{RRC} state machine with two states: \textit{CONNECTED} and \textit{IDLE} \cite{Huang2012}. With no active transmission, the \ac{UE} is in the low power \ti{IDLE} state where no radio resource is assigned.  Once a packet arrives, the  \ac{UE} jumps to the high power \ti{CONNECTED} state, in which radio resources are assigned and data transmission takes place.  To reduce the incumbent delay and energy consumption associated with the state promotion, the \ac{UE} waits for a duration called tail time in the \ti{CONNECTED} state before returning to the \ti{IDLE} state even after packet transmission is over. To save energy in the tail period, LTE uses  \ac{DRX} during which the cellular interface periodically monitors the control channel for incoming packets and then goes to sleep ~\cite{Huang2012}. Evidently, if video is downloaded during poor connection quality, then the smartphone will have a longer \ti{CONNECTED} state dwell time resulting in higher energy consumption. Existing \ac{ABR} video streaming algorithms, however, primarily focus on improving \ac{QoE} while paying little attention to energy savings.


\noindent \textbf{Improving QoE:}
\ac{ABR} video streaming algorithms either choose buffer occupancy~\cite{Huang2014,Spiteri2016}  or both buffer occupancy and current chunk or network throughput~\cite{yin2015control,Jiang2014,Sengupta2018,Xu2015,Mehr2019} to select optimal bitrates for future video chunks. Examples would be BOLA \cite{Spiteri2016} and MPC \cite{yin2015control}, respectively.
Pensieve \cite{mao2017neural} uses a deep RL algorithm for optimal bitrate selection to maximize over a \ac{QoE} metric. However, none of these  works focus on saving device energy consumption under mobility conditions in \ac{4G} LTE networks. \\
\indent In this work, we aim to improve video user's energy consumption over cellular networks while not compromising on \ac{QoE} by tuning playback buffer size to network throughput. Hence, the proposed algorithm should use cellular network throughput prediction.
Several works focus on bandwidth prediction for improving the bitrate selection of  ABR streaming algorithms~\cite{Bentaleb2019,Raca2019,Raca2018_2,yue2018linkforecast}. Some of these works also focus on predicting cellular network throughput~\cite{Raca2019,yue2018linkforecast,Raca2017,Raca2018_2,Raca2018_3,Samba2017, Ghasemi2018}. However, unlike our EnDASH algorithm, none of the works consider the unique situation of co-existence of different technologies and frequent handover from one technology to another for throughput prediction.


\noindent\textbf{Reducing Energy Consumption:} Several works in literature investigate energy consumption reduction of mobile phones independently of QoE or ABR streaming algorithms. The BarTendr algorithm in \cite{Schulman2010}  tunes the download sessions in 3G networks to the network conditions for saving energy. However, it quantifies the network condition using received signal strength only while giving no weightage to handovers or associated technologies. GreenTube in \cite{Xin2012} proposes to tune cache management to user behaviour and network conditions. A popular method to reduce energy consumption in mobile phones is to optimize the tail energy, which is achieved in ~\cite{Yang2018} by either prefetching or delaying packets.

\indent To tune packet downloads to network conditions so as to save energy requires a detailed energy profiling of the phones and service providers. This can be obtained through detailed measurement studies as in \cite{Huang2012}. While~\cite{Zhang2018M,Zhang2016,Zhang2016DASH,Khokar2019} focus on the measurement of power consumption of video traffic over HTTP in \ac{4G} networks, the effect of  mobility on signal strength  in \ac{4G} networks is presented briefly. In~\cite{Huang2012, Deng2018} are presented  measurement studies on mobility support in \ac{4G} \ac{LTE} networks. However, to the best of our knowledge, there is no comprehensive measurement study on video streaming under mobility in cellular-only networks. Furthermore, an inherent assumption in these papers is the uninterrupted availability of 4G signal. In contrast, the present work focuses on optimization of energy consumption and \ac{QoE} of mobile video users in scenarios where legacy networks are present in addition to 4G, under mobility conditions.