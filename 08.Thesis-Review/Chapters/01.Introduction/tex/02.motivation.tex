\section{Motivation of the Research}
The motivation behind this research comes from different facts, as we summarize below. 

\subsection{ABR Algorithms and Their Interplay with Computer Networks}
\Ac{ABR} algorithms are an essential part of any adaptive video streaming system, as discussed above. It is assumed that \ac{ABR} algorithms treat the streaming and other types of Internet traffic fairly in most cases. However, during the COVID-19 pandemic, various video streaming services started dropping the highest quality of video from their service to preserve the bandwidth available to the \acp{ISP}, as many people began working from home. At that time, the research community started questioning whether it was a fault of the video streaming system, or whether or not the existing \ac{ABR} algorithms adjust themselves to the lower bitrates when the load is high or something wrong with the deployment of those streaming services. While we do not know the answer, it is clear that the \ac{ABR} algorithms are fundamental, and the video streaming systems need to adapt to various network conditions. Questions like these motivate us to research the current commercial video streaming systems to explore how fairly they work over today's Internet.

Google's YouTube is one of the pioneers of video streaming over \acr{HTTP(S)} and a major online video streaming player. It serves a billion videos to 30 million users daily\footnote{\url{https://blog.youtube/press/} (\lastaccessedtoday)}. To scale the service to support these huge user-bases, YouTube has developed several improvements that involve content delivery networks, transport protocol, and adaptation algorithms. There is no doubt that YouTube gained the attention of the research community to understand the system. In the past, several pieces of research have studied YouTube for a better understanding of its functionalities and its impact on Internet traffic. However, most of the work either explore the traffic patterns and \ac{QoE} properties of YouTube~\cite{gill2007youtube,krishnappa2013dashing,wamser2016modeling,wamser2015poster,6757893ieeeexp,7129790ieeeexp} or explore the trade-off between video quality and data wastage~\cite{sieber2015cost,seufert2015youtube,sieber2016sacrificing}. Although these studies give an important aspect of YouTube, they failed to explore the internal parameter and streaming strategies. It is crucial to know the adaptation strategies and internal parameters of YouTube to improve the DASH-based system.

On the other side, the web being Google's primary business, it started a project called \textit{`Make the Web Faster'}\footnote{\url{https://developers.google.com/speed} (\lastaccessedtoday)}, where it developed several tools and optimization to load a web page faster. \acfi{QUIC}\cite{langley2017quic} is a product of this project where Google ditched the underlying \ac{TCP} and provided a new transport protocol based on the \ac{UDP}. Starting from the release of \ac{QUIC}, researchers have studied the performance of it in various scenarios, including \ac{DASH}-based video streaming. As \ac{QUIC} works differently than \ac{TCP}, and most of the \ac{ABR} algorithms used in different video streaming platforms are designed for \ac{TCP} as the transport protocol, those \ac{ABR} algorithms might not work well with \ac{QUIC}. Research works like \cite{bhat2018improving,van2018empirical} studied the impact of \ac{QUIC} over various \ac{ABR} algorithm and found that buffer-based \ac{ABR} algorithms are not well suited for \ac{QUIC}. However, those works are performed using old \ac{ABR} algorithms, and the studies are mostly simulation-based. Also, these works do not provide any insight into the root cause of the differences.

\subsection{Energy Efficiency in Online Video Streaming}
As per YouTube's report, users love to play online videos on a smartphone, which is also true for other streaming platforms. However, video streaming is inherently an energy-consuming service. The uneven distribution of \ac{LTE} cellular network base stations, clustered crowds, and non-optimized video app can make these things even worse. Research in \cite{10.1145/2910018.2910656} shows plenty of places to save energy by optimally configuring the streaming parameters like segment length, maximum buffer length, and segment download scheduling. Such research indicates that the interplay between the cellular network and the device's radio controller plays a significant role in the device's energy efficiency during online adaptive video streaming. It is important to learn various cellular network parameters and make informed adaptive decisions based on the learned parameters to make the adaptation more energy efficient.

\subsection{Live Streaming and Its Impact}
Live streaming is gradually becoming an alternative to \ac{TV} as most mega-events' live videos are broadcasted over the Internet using online live streaming. As more users shift towards online live streaming, it becomes difficult to scale such mega-events' streaming. The existing literature proposes solutions using \ac{P2P} networks to scale the live video streaming as players are in sync. However, \ac{P2P} based live streaming suffers as most of the \acp{ISP} allocate significantly less uplink bandwidth than downlink bandwidth to their subscribers. 

When a \ac{P2P} network spans over the Internet, traffic needs to go through different \acp{ISP} and \acp{AS}, and this traffic will count towards the user's data connection. So, users may not take it lightly. However, in mega-events, many users stay in clusters and share a common Internet backbone. An intelligently developed service can try to exploit the internal network to extend the service and provide better quality with lower overhead to the streaming server and the ISP. This gives plenty of scopes to optimize the live streaming systems for \ac{ABR}-based streaming.