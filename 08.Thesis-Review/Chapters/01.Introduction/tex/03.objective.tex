\section{Objectives of the Thesis}
The objectives of this thesis are as follows. 

\subsection{Understanding Adaptive Streaming Services over Today's Internet}
As we mentioned, almost all the \ac{OTT} streaming media services over the Internet has adopted \ac{DASH} and use some form of \ac{ABR} algorithms at the client application. Given the context and importance of video streaming services, it is necessary to understand how they work and how they interact with the web and other traffic over the Internet. It is also essential to understand the interactions between the underlying networking protocols, like \ac{QUIC}, and the \ac{ABR} streaming services running as Internet applications. With these requirements in mind, this thesis's first objective is to analyze YouTube streaming media services to explore the \ac{ABR} algorithms and how it interacts with the various networking protocols. Although the adaptation mechanism in YouTube is based on \ac{DASH}, we want to find the differences between the conventional \ac{DASH} principles and the streaming algorithm used by YouTube. YouTube is a closed source commercial \ac{OTT} application and uses proprietary services; the only way to understand the \ac{ABR} mechanism of YouTube is by reverse-engineering the system based on observations of its behavior. 

\subsection{Explore the Impact of Network Protocols and Client Mobility on the Performance of ABR Techniques}
The \ac{ABR} algorithms' performance depends on the observed network quality; however, it highly depends on the underlying transport layer, \ac{TCP}, until Google developed \ac{QUIC}. Our next objective is to explore the impact of the underlying transport protocols, such as \ac{TCP} and \ac{QUIC}, on the behavior of various modern \ac{ABR} techniques~\cite{Spiteri2016,mao2017neural}. We further envision exploring \ac{ABR} streaming algorithms' energy-efficiency, as a majority of the popular \ac{OTT} media services run as a client application over smartphones, and users prefer to watch videos while in transit. Consequently, the impact of mobility on smartphone energy-efficiency due to the \ac{ABR} algorithms' network traffic generation patterns is an essential field of study. Our objective is to explore the same over diverse networking environments.  


\subsection{Energy Efficient Video Streaming over Smartphones}
While \ac{QoE} is the primary goal for any \ac{ABR} algorithms, as mentioned before, looking into secondary objectives such as energy efficiency during video streaming is also essential when the services run over devices like a smartphone. Power management in a smartphone is crucial, which is also important from the perspective of video streaming, especially when users are in mobility. So, we aim to develop an \ac{ABR} algorithm, especially for smartphone-based video streaming, to reduce battery consumption while maintaining reasonable \ac{QoE}. We aim to utilize the radio-related information available in a smartphone, apart from various other information such as \acr{GPS} location, device speed, battery information, etc. This primary objective can be subdivided into multiple goals as follows.
\begin{enumerate}
	\item To find the correlation between radio, battery, and other physical parameters of a device with the network quality.
	\item To utilize these parameters to develop an \ac{ABR} algorithm to reduce energy consumption while video streaming.
	\item To provide reasonable \ac{QoE} based on the device setting, although it might sacrifice the finest \ac{QoE} to preserve energy.
\end{enumerate}

\subsection{Efficient ABR for Live Streaming}
This work aims to reduce Internet and server load while maintaining high \ac{QoE} during live streaming by exploiting the feature of synchronous playback and locality information within \ac{ABR}. Our objective is to design a live streaming player that can find and share video segments with other players in the locality by dynamically forming a \ac{P2P} network during the online video streaming. The challenge here is to maintain the synchronous playback among the players through \ac{ABR} streaming, as different players download different video segments and share them over the \ac{P2P} network. To support \ac{ABR} over such an architecture, the players need to decide the playback bitrate for the next video segment collectively and also have to schedule the segment downloads among the players by maintaining fairness in the \ac{P2P} network. \ac{QoE} optimization needs to be the overall goal for this collective \ac{ABR} decision; however, each player needs to coordinate with others to achieve the same. 

