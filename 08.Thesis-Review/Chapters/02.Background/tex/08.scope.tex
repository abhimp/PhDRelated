\section{Summary and Open Scopes}
The online video streaming system is a vast and complex system to provide the best \ac{QoE} for all the users and efficient energy consumption for smartphone-based users. In this chapter, we discussed various research works and studies to improve the video streaming system. Now we summarize them and discuss the scope of this topic.

\subsection{Analysis of Existing Commercial Streaming Services: Case Study of YouTube}
YouTube started as a video sharing service in 2005. Later, it was acquired by Google. At the inception, YouTube used the Adobe Flash Player plugin to play video in the browser. Although YouTube played video in the browsers only using the Flash Video format, it allowed users to upload the video in various formats, including WMV, MPEG, AVI. YouTube also exploited the progressive download feature from Adobe Flash Player to play the video with partially downloaded file~\cite{gill2007youtube}. After the HTML5 standard has launched, YouTube started using \ac{HTML5} embedded video player as an experimental version in January 2010. From that time, YouTube started using the fixed resolution (i.e., non-adaptive) video playback with an option to change the video quality manually. However, in 2013, YouTube started the trial on \ac{DASH}-based streaming and made it the default playback mechanism in 2015\footnote{\url{https://arstechnica.com/gadgets/2015/01/youtube-declares-html5-video-ready-for-primetime-makes-it-default} (\lastaccessedtoday)}.

Existing studies on YouTube video streaming system and video \ac{QoE} can be grouped into two broad classes. The first class of works explore traffic patterns and video \ac{QoE} properties of YouTube~\cite{gill2007youtube,krishnappa2013dashing,wamser2016modeling,wamser2015poster,6757893ieeeexp,7129790ieeeexp}. These works mostly study YouTube's behavior at the periphery, which, although provides a summary of performance metrics, fails to say much about the internals of YouTube's video streaming protocol. The second class of studies, however, explore the adaptive streaming characteristics of YouTube. In \cite{finamore2011youtube}, the authors investigated YouTube's data delivery system from the end-user view. They illustrated evidence of massive wastage of downloaded data since viewers often do not watch entire videos -- the study, however, was performed at a time when YouTube used progressive download as the streaming mechanism and is therefore stale. \cite{krishnappa2013dashing} is probably the first work to evaluate YouTube's performance since its adoption of adaptive streaming -- the authors claim that YouTube gains $83\%$-$95\%$ in terms of bandwidth by switching from progressive download to DASH. Some recent works~\cite{sieber2015cost,seufert2015youtube,sieber2016sacrificing} have studied YouTube's DASH behavior to analyze the trade-off between quality and data wastage -- however, their approximations lead to gross overestimation. They perform controlled experiments by varying the underlying link bandwidth and compute wastage.

\subsection{Impact of QUIC over DASH}
\ac{DASH} or \acr{DASH}-like video streaming systems use \ac{HTTP(S)} protocol to fetch content from the server. While \ac{HTTP(S)} is widely accepted, it uses \ac{TCP} as the underlying protocol. So, the performance of \ac{DASH} based video streaming system is highly dependent on the performance of \ac{TCP}. Although \ac{TCP}'s performance is the best for long running flows, it does not work well with short flows. Furthermore, most of the \ac{ABR} system produce ON-OFF traffic for \ac{DASH} based streaming system. ON-OFF traffics are essentially short flows. So, \ac{DASH} can not perform at its peak due to this effect. To avoid such a problem, Google developed a new transport protocol \ac{QUIC}~\cite{langley2017quic} which works on top of \ac{UDP} and provides an interface for \ac{HTTPS}. According to Google, \ac{QUIC} reduces rebuffering up to 15\% for mobile users.

Various recent studies \cite{Biswal2016,Megyesi2016,bhat2017not} have revealed that \ac{QUIC} can improve web performance by reducing the page load time even at poor network conditions. Among them, a few works have explored the adaptive streaming performance over \ac{QUIC}. In~\cite{bhat2017not}, the authors have empirically shown that \ac{DASH} suffers over \ac{QUIC}. Although they have given the first indication that the current \ac{ABR} techniques might not perform well over \ac{QUIC}, their analysis is mostly focused on buffer-based ABR and does not look into various QoE metrics as explored in the recent literature~\cite{yin2015control,mao2017neural}. In a follow-up work~\cite{bhat2018improving}, the authors have explored the \ac{QUIC} retransmissions to improve the buffer-based \ac{ABR} over \ac{DASH}. In~\cite{van2018empirical}, the authors have used an emulated setup to analyze the buffer-based \ac{ABR} techniques over \ac{QUIC} and proposed a \ac{QoE} prediction mechanism for adaptive streaming over \ac{QUIC}. Also, these existing studies have indicated that \ac{QUIC} might not suit well for buffer-based \ac{ABR}. They have not explored the performance of advanced \ac{ABR} techniques over \ac{QUIC}, although it is important as \ac{QUIC} is gaining popularity and has become a choice of protocol for most web services. These reasons led us to believe that there is a requirement for an in-depth study on the performance of different advanced \ac{ABR} algorithm on top of \ac{QUIC}.

\subsection{Smartphone Energy Consumption and DASH based Video Streaming}
Power consumption is a concern with all mobile devices, and smartphones are no different. While most of the services like cellular, BlueTooth, push notifications are designed to consume less energy to provide long battery backup. However, online video streaming services are not part of those services, and it is challenging to improve significant energy efficiency. As we discussed before, the existing work has just scratch the surface of the energy-efficient solution for video streaming services. So, there are scopes to improve the energy-efficiency of online video services.

Although existing studies have focused on the \ac{QoE} aspects during video streaming, they remained silent about their effectiveness in terms of energy-efficiency over a smartphone or similar platforms. The recent \ac{ML}-based \ac{ABR} algorithms are likely to be more power-hungry compared to the classical \ac{ABR} techniques, therefore, a thorough analysis of the energy-consumption behavior of the existing \ac{ABR} techniques is an important requirement in the current context.

\subsection{DASH and Live Video Streaming Systems}
Online live video streaming has become an alternative to the live \ac{TV} broadcast. Online live streaming services stream the big and important events and also allow individual users to go live and broadcast videos. \ac{DASH} supports live streaming as the playback technique is the same for both live and \ac{VoD} streaming. However, the server-side part is not the same, and there are several challenges to serve live streaming to a very large audience from all over the globe. It is very difficult to serve live stream to millions of users even with \ac{CDN}. Live streaming providers are trying to exploit live streaming's synchronous behavior (i.e., all the players play almost the same portion of the video) and extend the \ac{CDN} using \ac{P2P} networks. While most of the work concentrates on routing \ac{P2P} traffic through \ac{ISP}, there is a scope to exploit the same for in campus networks (i.e., lot player connected via private network and common Internet backbone). However, designing an \ac{ABR} technique on top of such architectures is tricky.

To summarize, we understand that although \ac{ABR}-based video streaming techniques have seen a lot of researches and investigations in recent times, there are still plenty of scopes for improving the performance of such systems. As millions of users over the Internet streams online videos, such optimizations can have significant impact on the commercial deployment of video streaming services. With these objectives in mind, we start exploring the performance of existing commercial \ac{VoD} applications that use \ac{ABR} at their core, and do a thorough study of the YouTube \ac{ABR} streaming services, as discussed in the next section.
