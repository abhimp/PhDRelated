\section{Adaptive Live Streaming}
\label{chapter02:live}
\ac{DASH}-like video streaming system is widely used as it is superior to other systems, and live video streaming is no exception. Although interactive video sessions are difficult to support via \ac{DASH} due to the rigid latency bound, non-interactive live video streamings are well supported. Static (or \ac{VoD}) and live video streaming are almost identical as per player perspective, except that live streaming does not support video seeking. However, providing live streaming is not easy as all the segments need to be captured, processed, and distributed to the nearest server before it can be served to the client. All these tasks need to be performed with reasonably small delay; otherwise, the user experience might degrade. Researchers have developed various systems to reduce the delay and scale of the live streaming to millions of users.

\subsection{Crowdsourced Live Streaming}
Live video streaming requires live transcoding into multiple different quality versions and different codecs to support a large set of platforms with higher \ac{QoE}. Unlike TV broadcasting, live streamings are mostly crowdsourced. The crowdsource streamer and viewers are geo-distributed and use heterogeneous platforms. The streaming service provider needs to support all those kinds of streamers and viewers and provide the best possible \ac{QoE}. Chen \etal\ explored the possibility of locating a suitable transcoding server as per the geolocation of streamer and viewer with the help of cloud federation in their research paper~\cite{7218642}.

Even though the cloud can be used to transcode live streaming effectively, it is not always economical as not all the videos are equally popular, and clients may not use all the quality and codec versions. However, transcoding in multiple codec and quality is expensive computationally as well as economically. So streaming providers need to be careful about the selection of the codec and quality level they support for each live streaming. Aparicio-Pardo \etal\ have designed an \ac{ILP}-based algorithm in their paper \cite{10.1145/2713168.2713177} to decide the number of quality levels required for live streaming, based on the streaming characteristics and popularity. It decides the \acsu{CPU} (\acsu{GPU}) budget for each live stream and decides the quality levels. 


\subsection{SmoothCache 2.0}
To reduce the distribution overhead, server load, and improve the quality of experience during the live streaming, SmoothCache~2.0~\cite{10.1145/2713168.2713182} provides a solution involving \ac{P2P} networking with \ac{DASH}. Roverso \etal\ have exploited the fact that all the live streaming players need to be in sync with very little tolerance $\delta$. The $\delta$ is the time when a player searches for the required segment in other peers and, if failed, fetches it from the \ac{CDN}. The authors have used optimizations like pro-active prefetching to reduce the overhead. Pro-active prefetching allows few players to fetch the segment as soon as they become available at the server.

\subsection{Neural-Enhanced Live Streaming (LiveNAS)}
\Ac{QoE} of a live stream is dependent on the network quality and computation capability, especially when random individual users start streaming using commodity devices like a smartphone with a cellular data connection. Due to the lack of dedicated Internet connection and lack of infrastructure, these users can not ensure a steady streaming quality. Kim \etal\ have designed LiveNAS~\cite{10.1145/3387514.3405856} to improve video quality in such cases. LiveNAS runs super-resolution to upgrade video quality at the ingest server if the uploaded video quality drops during the playback.

\subsection{Summary}
Live video streaming is apparently similar to the \ac{VoD} video streaming system. However, live streaming needs server side support for live content generation and distribution. This task is fairly complex if a streaming provider wants to support crowdsourced live streaming. They have to allocate resources to encode and distribute the video content. On the other hand, live streaming of mega-event attracts millions of viewers globally. Existing literature tries to scale the distribution network by utilizing \ac{P2P} networks. Work like \cite{10.1145/2713168.2713182} have tried to utilize the peer-to-peer network to share video segments among players. However, it may not work well if players have to use the Internet uplink to share the video segment, as uplink speeds are usually low and count towards their data cap. We believe that there is scope to improve live streaming as many live streaming video players share the same Internet connections.