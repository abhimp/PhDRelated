\begin{comment}
\subsection{Related Work}
Various recent studies \cite{Biswal2016,Megyesi2016} have revealed that QUIC can improve the web performance by reducing the page load time even at poor network conditions. Among them, a few works have explored the adaptive streaming performance over QUIC. In~\cite{bhat2017not}, the authors have empirically shown that DASH suffers over QUIC. Although they have given the first indication that the current ABR techniques might not perform well over QUIC, their analysis is mostly focused on buffer-based ABR and does not look into various QoE metrics as explored in the recent literature~\cite{yin2015control,mao2017neural}. In a follow-up work~\cite{bhat2018improving}, the authors have explored the QUIC retransmissions to improve the buffer-based ABR over DASH. In~\cite{van2018empirical}, the authors have used an emulated setup to analyze the buffer-based ABR techniques over QUIC and also proposed a QoE prediction mechanism for adaptive streaming over QUIC. Also, these existing studies have indicated that QUIC might not suit well for buffer-based ABR. They have not explored the performance of advanced ABR techniques over QUIC. We argue that there is a requirement to analyze the recent ABR techniques like MPC~\cite{yin2015control} and Pensieve~\cite{mao2017neural} over QUIC because these recent studies have indicated that buffer-based techniques are aggressive towards video-bitrate maximization whereas suffers in terms of playback smoothness and rebuffering. To the best of our knowledge, this is the first study that explores the performance of advanced ABR techniques over QUIC as the end-to-end transport protocol.

\end{comment}
