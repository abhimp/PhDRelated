\section{Introduction}
With the pervasive penetration of smartphones, cheap mobile data plans, and the development of video streaming over HTTP, video steaming has become one of the most sought-after services over the mobile Internet. The availability of applications to record and publishing video as well as the video editing application, people are sharing a large number of videos. These platforms also allow people to create video content in the regional language and accents, which allow more people to watch those videos as it is easy to understand. In these scenarios, end-user watch videos whenever they get time even during the time of commute to or from work and get entertained.

The experience of watching videos is essential for the user as well as the video provider, no matter where they are watching the video. However, it is pretty challenging to maintain the quality of experience (QoE). At the same time, users are mobile as network quality varies due to various reasons, including the uneven distribution of cell towers, no of users per tower, and many more. While network service providers are trying to increase network quality, online video providers are also trying to improve the quality of experience by tweaking the streaming system.

Dynamic Adaptive Streaming over HTTP (DASH) is a video streaming system used by different applications and video streaming providers. Although there are several other video streaming systems developed by different organizations, they are similar to DASH. In the DASH based video streaming system, first, the video needs to be encoded in different quality variants. After that, those video files are segmented in equal length (playback duration) video chunks and stored in different files. Here videos are encoded in such a way so that a video player can play any segment independently. A DASH video player observes the network condition and decide the quality level accordingly and download the segment. The algorithm selects the video quality for the next segment call adaptive bitrate (ABR) algorithm.

While ABR algorithms are very vital to provide a better quality of experience, popular video streaming provider tries to keep the content closer to viewers so that the video content can be fetched quickly without overloading the Internet. The content delivery networks (CDNs) are here to help from these types of problems. CDNs are composed of few origin servers which store all the content and a massive number of geographically positioned frontend servers. Whenever clients request content, the request directed to the nearest frontend server. The frontend server, in turn, contact one of the origin servers for the content and server it to the client. To reduce the load of the origin server as well as to reduce the traffic in CDN WAN, the frontend server tends to catch the content it serves. As the frontend servers have limited cache size, it needs an optimal cache policy to store and remove content.

In the case of video streaming with CDN, two components are vital. These are a) the cache policy and b) the ABR algorithm. If the goodput at the application level varies due to the CDN cache, any ABR algorithm can not perform optimally. However, as the CDN caches are not aware of the ABR algorithm, it can't prefetch correctly to reduce the latency. Similarly, ABR algorithms run at the client end, which is unaware of the cache policy or cache state; it is complicated to make an informed decision.

Despite the problem, an ABR algorithm is very vital for the quality of experience. Finding out the best or optimal ABR algorithm is a hot research topic around the globe. There are several ABR algorithms developed over the years, for example, BOLA\cite{dash:bola}, MPC\cite{dash:mpc}, Pensieve\cite{dash:pensieve}, Oboe\cite{dash:oboe}, and HotDASH\cite{dash:hotdash}. Although these algorithms improve the QoE, they do not consider the deployability of the algorithms.

These algorithms need a massive amount of computing power, and most of the algorithms have several library dependencies which might not be available in all the platforms, especially in smartphone and browsers. Most of the time, it also needs a trained model to get optimal performance, which might need to change for each video and several MBs in size. As DASH like system uses a dumb server and smart client, the algorithm runs in the client or the player itself; it is not feasible to run an ML-based ABR algorithm at the end-user devices. 

Although authors of these papers sometimes provide a demo to run their respective ABR algorithms in browser with standard DASH based implementation like dash.js, those techniques are not scalable. For example, authors of the Pensieve run an ABR server in the client system and connect them from browsers javascript. It works, but it does not scale as everyone needs to run a local server. To overcome this problem, we have developed a streaming system called Split-DASH. We discuss this architecture in detail in section \ref{sec:Split_DASH_architecture}. 
