\section{Related Work}
Various existing researches have focused on improving QoE of the adaptive streaming protocols over HTTP(S). Huang \etal proposed a buffer-based adaptation algorithm~\cite{buffer-based-sigcomm-2014}, which has been extended later from different directions, such as based on playback buffer monitoring (BOLA)~\cite{bola2-acm-mmsys2018}, control theoretic approach (MPC)~\cite{MPC-SIGCOMM-2015}, reinforcement learning based approach (Pensieve~\cite{Pensieve}), and so on. However, these mechanisms are primarily developed for standalone players, where players' connection quality vary over time. Although BOLA, MPC and Pensieve improve the QoE for end user significantly, nevertheless, the QoE for a single player is limited by the network condition; consequently researchers have explore peer assisted video streaming to utilize collective download capability of the playback devices. In this direction, 
%Therefore, we move towards a collaborative streaming approach. 
Liu \etal have measured the performance bound for peer-assisted streaming over a tree-based network~\cite{Liu-sigmetrics-2008}. 
%They have considered a tree-based peer-to-peer network. I
Various other works~\cite{Wang:ACMmm-2011,TNET-Coop-2015,EdgeNode-GameTheory-Globecomm-2018, Collaborative-P2P-WCNC-2018} have explored possibilities and strategies of peer-based video streaming. Wang \etal proposed a peer-based solution for live streaming, which reduces the source peer load \cite{Wang:ACMmm-2011}. Stefan \etal proposed a solution for peer-assisted DASH based video streaming~\cite{P2PHttp-2012}, where they cluster players with similar download speed on the Internet to collaborate in the streaming. 
%They need to change the DASH MPD files according to the peer network as MPD file itself contain the peer addresses. 
Ishakian \etal proposed a peer-assisted cloud-based streaming system called AngelCast~\cite{ISHAKIAN201714}, where they achieved significant scalability with the available clients' upstream capability.
%Feng \etal advocated cloud assisted live media streaming in \cite{TNET-Migration-2016}. 
%Authors from \cite{PeerAss-RTMFP-2018, EdgeNode-GameTheory-Globecomm-2018, Collaborative-P2P-WCNC-2018} and references therein have discovered the possibility of peer-assisted video streaming. 
%Zou \etal argument the possibility of using UDP based RTMFP for PPTV~\cite{PeerAss-RTMFP-2018}. They found that their proposed architecture improves the quality and reduces the server load.
Edge assisted video streaming with coalition formation have been explored in \cite{EdgeNode-GameTheory-Globecomm-2018}, where a game theoretic formulation has been used for coalition formation. 
%All these works explored the possibilities of peer-to-peer or edge assisted media streaming. 
However, the existing works primarily consider that the peers are connected with the Internet and they have bounded by the upstream capability. In contrast to these existing schemes, we explore the possibility of streaming live video sharing when players are connected via an unconstrained local area network (LAN), where the network quality of the peer nodes can change over time, and therefore, a collaborative adaptive streaming strategy needs to be designed. 
%However, the players either have different types of Internet subscription, or they have a shared Internet connection.


% https://doi.org/10.1145/1375457.1375493
% https://doi.org/10.1145/2072298.2071982
% https://doi.org/10.1109/PV.2012.6229730
% https://doi.org/10.1109/TPDS.2009.167
% https://doi.org/10.1109/TCSVT.2016.2601962
% https://doi.org/10.1109/GLOCOMW.2018.8644422
% https://doi.org/10.1109/WCNC.2018.8377226
% https://doi.org/10.1016/j.comcom.2017.06.011
% https://doi.org/10.1109/TNET.2014.2362541
% https://doi.org/10.1109/TNET.2014.2346077