\section{\textbf{Conclusion}}\label{section:conclusion}
%In many developing countries, 4G coverage is not ubiquitous. As a result users often have to fallback to legacy networks that support lower throughput, resulting in a higher energy consumption and lower QoE. 
\acresetall
In this report, we propose EnDASH -- an energy aware ABR video streaming algorithm, which minimizes energy consumption while not compromising on QoE of users under mobility. It exploits the high throughput regions in the user's trajectory for aggressive fetching of video chunks thereby reducing energy consumption even in regions with limited 4G coverage and significant presence of legacy networks.  To achieve this, it intelligently tunes the playback buffer length with the average predicted throughput and then resorts to optimal bitrate selection for video chunks. As a result, the buffer length increases sometimes; although, the consequent cost  escalation is negligible. This is because the cost is incurred due to  (a) extra memory usage which is cheap  and can be ignored, and (b) aggressive fetching which may lead to some wastage, but such wastage is rare and minimal, therefore, hardly having any cost impact.\\  %Moreover, the data tariff is extremely cheap ; thus hardly having any impact on the cost.\\
\indent EnDASH predicts the cellular network throughput using Random Forest Learning. It tunes the buffer length and selects the optimal chunk bitrates using Reinforcement learning. EnDASH is able to improve the maximum energy consumption by about 30.4\% in comparison to the popular Pensieve algorithm although with reduction in QoE. Since EnDASH is a tunable algorithm it can be designed to adapt to a specific  requirement, such as energy or QoE -- this would be our immediate future work.  Additionally, our future work will also involve real-life implementation of EnDASH and investigating the corresponding improvement in energy efficiency.
