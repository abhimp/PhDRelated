\section{Conclusion}
%The QUIC is a new protocol developed the Google to {\it make web faster}. The Google claims that it performs better than traditional TCP based HTTP/HTTPS connection. Several existing work shows that QUIC performance is comparable to TCP. In this work we have compared QUIC's permance for web based video streaming. Our result shows that QUIC's performance is slightly poorer than the existing TCP in terms of quality of experience.
This letter gives an analysis of the recent advanced ABR techniques over the QUIC as the end-to-end transport protocol. We observed that all the ABR techniques are sensitive to sudden increase or drop in the client-perceived link bandwidth, and therefore are more compatible with TCP rather than QUIC. \blue{The QUIC multiplexing of audio and video streams over a single UDP socket results in additional response latency for the audio segments, which are not captured during the calculation of channel throughput. As a consequence, the ABR algorithms take incorrect decisions during selecting the bitrates based on the calculated throughput over a QUIC connection.} The analysis discussed in this letter opens up a new direction of research on exploring ABR techniques over QUIC which is the de-factor transport protocol for Google services. 