\section{Introduction}
With pervasive penetration of smartphone, cheap Internet and development of video streaming over HTTP, video steaming became one of the most important service of the Internet. The availability of application to record and publishing video as well as the video editing application, a large number of videos are being shared by people. These platforms also allows people to create video contents in the regional language and accents which allow more people watch those videos as it is easy to understand. In this scenarios, end use watch video whenever they get time. Smartphone user watches video during commute to or from the work and get entertained.

The experience of watching very important for the user as well as the video provider no matter where they are watching the video. However, it is pretty difficult to maintain quality of experience while users are mobile as network quality varies due to various reasons including uneven distribution of cell towers, no of users per tower and many more. While network service providers are trying to increase the network quality, online video providers are also trying to improve quality of experience by tweaking the streaming system.

Dynamic Adaptive Streaming over HTTP (DASH) is a video streaming system used by different applications and video streaming provide. Although there are several other video streaming system developed by different organization, they are similar to DASH. In DASH based video streaming system, first the video need to be encoded in different quality variants. After that, those video files need to be segmented in equal length (playback duration) video files. Here videos are encoded in such a way so that a video player can play any segment independently. A DASH video player observe the network condition and decide the quality level accordingly and download the segment. The algorithm used to select the video quality for next segment call adaptive bitrate (ABR) algorithm.

The ABR algorithm is very vital for quality of experience. Finding out the best or optimal ABR algorithm is a hot research topic around the globe. There are several ABR algorithm developed over the years for example BOLA, MPC, Pensieve and HotDASH. Although these algorithms improve the QoE, they do not consider the deployablity of the algorithms.

In the recent years, there are lot developments in the area to adaptive bitrate algorithms. Some of those algorithms are based on different machine learning techniques like Penseive, HotDASH. These algorithms need powerful system to run and most of the algorithms have several library dependency which might not available is all the platform, specially smartphone and browsers. Most of the time it also need a trained model to get the optimal performance. As DASH like system uses a dumb server server and smart client, the algorithm need to be run in the client or the player it self. However, it is not feasible to run ML based ABR algorithm in the end user devices. 

Authors of this paper sometime provides a demo to run those algorithm in browser, but that technique is not scalable. For example authors of the pensieve run a ABR server in the client system and connect them from browsers javascript. It works, but it does not scale as every one need to run a local server. To overcome this problem we have developed a streaming system called Split-DASH. We discuss this architecture in details in section \ref{sec:Split_DASH_architecture}. 
