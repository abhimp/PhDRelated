

\section{Introduction}
The Internet that connects all the devices in globe, was designed almost half century ago with basic goal of connectivity. However the use cases, services and users have been grown several magnitude in last one-or-two decades. Few of the major use case now a days are social network and video streaming. The increased rate of mobile phone users increased the network traffic related to social networking and video streaming to few magnitude. Also, now a days we have huge amount of device-to-device communication due the explosion of IoT and connected sensor. However, the Internet was not designed to handle these type of traffic. Researchers around the globe are trying scale the Internet to support the use cases.

The Internet and almost all computer networks follow the conceptual design from Internet protocol suite which is commonly known as TCP/IP protocol suite. In TCP/IP protocol suite network layer protocol \acrfull{ip} and transport layer protocol \acrfull{tcp} are the main component. \acrshort{ip} provides connectivity between remote devices and \acrshort{tcp} provided the ability to transfer data between two remote processes. \acrshort{tcp} also provides end to end congestion control and flow control. However, The nature of data traffic in todays' Internet is different from the one that was envisioned during the development of the TCP/IP protocol suites~\cite{rexford2010future}.

\textbf{Limitations of current transport protocols}: While TCP provides reliable end to end connectivity,  it has three fundamental problems.  

(a) \textit{Mobility:} Supporting seamless communication during mobility is one of the major requirements for today's Internet. However TCP is a connection oriented protocol; a connection in TCP is identified by the tuple (source IP, destination IP, source port, destination port). If one of the source or destination IP changes, the ongoing TCP connection needs to be dropped, and a new connection needs to be established~\cite{yadav2016msocket}. 

(b) \textit{Multihoming:} If a device is connected to the Internet via multiple network interfaces, it is called a multihomed device. Multihomed devices have the power to transmit or receive data via multiple paths through multiple physical interfaces. However, normal TCP cannot utilize multiple interfaces available to a multihomed device~\cite{de2016observing}. 

(c) \textit{Short-lived Flows:} TCP employs a slow start phase to handle network congestion. However, a short-lived flow may fail to come out of the TCP slow start phase, resulting in underutilization of available network resources~\cite{de2016throughput}. Interestingly, many of the today's application over the Internet, like web browsing, IoT communication, many smartphone apps, generate parallel as well as sequential short lived flows. 

%\noteam{Need to write gist on related work}
To address one or more of these issue, researchers have developed different transport protocol like MPTCP, QUIC, MPQUIC. However those protocol have limitations. MPTCP solves the problem of using multiple interface simultaneously while it suffers from short lived connections. Other protocol like QUIC and MPQUIC can not access interfaces directly which limits their capability of using multiple interface. Although MPTCP supports mobility, QUIC does not.

\textbf{Our Contributions}:
As a consequence, in this report, we develop a multi-path protocol, called {\em Viscous}, that can address the three fundamental shortcomings of the existing end-to-end protocols -- mobility, multi-homing and short-lived event driven flows. We develop Viscous as a user level flow management protocol that runs on top of the UDP protocol, similar to Google's QUIC, however is bundled with a number of new features. Viscous can multiplex flows from multiple applications and decouples flow control from congestion control, so that an application flow does not suffer from network bandwidth underutilization problem. Further, Viscous is free from HOL blocking, and supports reliability as well as congestion control over unreliable UDP based end-to-end data transmission. We implement Viscous as a Linux application library, and test its performance over an emulated and physical platforms. We observe that Viscous can significantly improve the transport layer protocol performance compared to standard TCP, MPTCP and QUIC, when large number of short lived flows are generated from the edge devices. We have also provided a algorithm to be used to give priority to one or more flows. This algorithm solves the proportional fairness problem. We have also provided the necessary proof that the proposed algorithm provides optimal solution for the problem.