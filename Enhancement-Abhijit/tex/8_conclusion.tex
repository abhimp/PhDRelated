\section{Conclusion}
Considering the limitations of current transmission protocols for today’s diverse device and traffic characteristics, in this paper, we have developed Viscous that provides a middleware between the user application and the transport layer to handle end-to-end network characteristics. Viscous is completely compatible with the current network protocol stack, while it provides significant performance boost in the application throughput. It is developed on top of the UDP along with a set of novel features to support mobility, multi-homing and short-lived parallel and sequential flows. From the prototype implementation and performance analysis over an emulated environment, we show that the worst case performance of Viscous is similar to TCP, which indicates its applicability for a wide range of applications. 
In future, our target is to enhance Viscous with application layer QoS support to make it an alternate for both non-real time and real time traffic transport.  


\section*{Publications}
[1] Abhijit Mondal, Satadal Sengupta, B.R. Reddy, M.J.V. Koundinya, Chander G., Pradipta De, Niloy Ganguly, Sandip Chakraborty, ``Candid with YouTube: Adaptive Streaming Behavior and Implications on Data Consumption'' in Proceedings of the 27th Workshop on Network and Operating Systems Support for Digital Audio and Video (NOSSDAV'17), Taipei, Taiwan, June 20 - 23, 2017 .

[2] Abhijit Mondal, Sourav Bhattacharjee, Sandip Chakraborty, ``Viscous: An End to End Protocol for Ubiquitous Communication Over Internet of Everything'' in Proceedings of 42nd Annual IEEE Conference on Local Computer Networks (LCN 2017), Singapore, October 9-12, 2017.

[3] Abhijit Mondal, Sandip Chakraborty, ``Viscous: Design, Development and Implementation of an end-to-end protocol over Heterogeneous Internet'', submitted to IEEE/ACM Transaction on Networking.
