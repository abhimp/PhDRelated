\chapter{Conclusion and Future Works}
The Internet has seen a huge boom in the video streaming traffic in recent days and more so after the COVID-19 pandemic when OTT media services and interactive streaming applications became the de facto both for spending times in leisure and day-to-day activities. It this worthwhile to mention that the OTT media giants like NetFlix, YouTube, Amazon, Apple, etc. have dropped the highest quality level from the available playback bitrates to save bandwidth in the Internet\footnote{\url{https://www.theverge.com/2020/3/20/21188072/amazon-prime-video-reduce-stream-quality-broadband-netflix-youtube-coronavirus
} (Access: \today)}. During this time one obvious question that got raised within the networking research community is that whether the ABR algorithms are themselves not capable of efficiently adjusting the bitrates based on the available capacity of the network. The study of existing ABR algorithms and design optimizations on top of that is thus still a very important area of research which needs significant attentions.  

Towards this direction, this thesis discussed four contributory chapters dealing with analyzing and optimizing video streaming services over the Internet. Our first contributory chapter deals with analyzing the YouTube ABR mechanism and with finding out various parameters that impact YouTube streaming at large. The second contributory chapter explores the impact of protocol choice and mobility on ABR performance. Based on these analyses, the third contributory chapter develops an energy-efficient streaming algorithm for the Internet. The final contributory chapter discusses the design of an adaptive live streaming platform by exploring the ALTO service. In a nutshell, this thesis enriches the current literature from two different aspects. First, it thoroughly analyzes the current ABR techniques over popular OTT media services like YouTube and opens up various problems and challenges associated with it. Second, it proposes two optimizations over the current streaming media protocols, one in the direction of supporting energy efficiency during VoD streaming over smartphones, and the other in developing an efficient mechanism for adaptive live streaming.

\section{Future Works}
Although this thesis solves a few challenges associated with scaling up video streaming services over the Internet, there are many open problems that can be taken up as future research directions in this area. We summarize them as follows. 

\subsection{Amalgamating DASH with QUIC: Do We Need a New ABR, or Do We Need Modifications in QUIC?}
We have shown the current ABR techniques do not work well with QUIC and highlighted the reasons behind that. Therefore, efficient protocols can build up the synergy among the application layer streaming services and the underlying network protocol. Second, the practical implementation of ML and DL-based ABR techniques is still an open problem. Necessary architectural changes need to be designed to support the ML and DL-based ABR techniques over the client devices. Third, innovative methods like super-resolution have been proposed in the literature. It will be interesting to check how these modern techniques work over the existing ABR algorithms. 