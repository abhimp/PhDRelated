\chapter{Conclusion and Future Works}
The Internet has seen a massive boom in the video streaming traffic in recent days and more so after the COVID-19 pandemic when \ac{OTT} media services and interactive streaming applications became the de facto both for spending times in leisure and day-to-day activities. It this worthwhile to mention that the \ac{OTT} media giants like NetFlix, YouTube, Amazon, Apple, etc. have dropped the highest quality level from the available playback bitrates to save bandwidth in the Internet\footnote{\url{https://www.theverge.com/2020/3/20/21188072/amazon-prime-video-reduce-stream-quality-broadband-netflix-youtube-coronavirus} (Access: \today)}. During this time, one obvious question that got raised within the networking research community is whether the \ac{ABR} algorithms can efficiently adjust the bitrates based on the available capacity of the network. The study of existing \ac{ABR} algorithms and design optimizations is still a crucial area of research that needs significant attention.  

Towards this direction, this thesis discussed four contributory chapters dealing with analyzing and optimizing video streaming services over the Internet. Our first contributory chapter deals with analyzing the YouTube \ac{ABR} mechanism and with finding out various parameters that impact YouTube streaming at large. The second contributory chapter explores the impact of protocol choice and mobility on \ac{ABR} performance. Based on these analyses, the third contributory chapter develops an energy-efficient streaming algorithm for the Internet. The final contributory chapter discusses the design of an adaptive live streaming platform by exploring the \ac{ALTO} service. In a nutshell, this thesis enriches the current literature from two different aspects. First, it thoroughly analyzes the current \ac{ABR} techniques over popular \ac{OTT} media services like YouTube and opens up various problems and challenges associated with it. Second, it proposes two optimizations over the current streaming media protocols, one in the direction of supporting energy efficiency during VoD streaming over smartphones, and the other in developing an efficient mechanism for adaptive live streaming.

\section{Future Works}
Although this thesis solves a few challenges associated with scaling up video streaming services over the Internet, there are many open problems that can be taken up as future research directions in this area. We summarize them as follows. 

\subsection{Amalgamating DASH with QUIC: Do We Need a New ABR, or Do We Need Modifications in QUIC?}
We have shown in Chapter~\ref{chapter03} that the current \ac{ABR} techniques do not work well with \ac{QUIC} and highlighted the reasons behind that. We observed that the multiplexing of multiple application flows over a single transport layer socket creates a problem when the flows send data at different rates. Consequently, the audio and the video channels for a streaming media service contend for the resource over a single socket buffer, resulting in such a problem. To solve such issues, we need further investigations both from \ac{QUIC}'s perspective and \ac{ABR}'s perspective. This can be an exciting aspect of future research. 

\subsection{ML-based ABRs: How Effective Are They in Real Deployments?}
Many recent ABR algorithms, such as Pensive~\cite{mao2017neural}, Oboe~\cite{Akhtar2018}, PREM~\cite{9155492}, etc. propose utilizing ML and DL for bitrate adaptation. However, the evaluation of those algorithms is mostly emulation-based; therefore, it is not clear how effective they are during real deployments, for example, over a smartphone. ML/DL mechanisms are likely to take more resources like \acsu{CPU} cycles, memory, power, etc., compared to conventional buffer-based, throughput-based, or hybrid algorithms. Therefore, it is necessary to implement those mechanisms as real applications and analyze their performance thoroughly.   

\subsection{Making ABR Robust with New Technologies}
Recently, various innovative methods, such as super-resolution~\cite{9155384}, have been proposed in the literature. It will be interesting to check how these modern techniques work over the existing ABR algorithms. Further, the impacts of such methods can be analyzed considering their implications both from the end-user's perspective and the service provider's perspective. 


In summary, while this thesis has explored a timely problem of scaling up video streaming services over the Internet, there are multiple open problems which are still needed to be investigated for making the network robust for \ac{OTT} media services. 