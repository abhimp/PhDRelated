%\section{Introduction}
Online video streaming is the most popular service on the Internet. The pervasive penetration of smartphones and the availability of cheap LTE networks make it even easier to reach millions of users. Video streaming can be primarily categorized into three categories: i) static video or video-on-demand (VoD) streaming, ii) live video streaming, and iii) interactive video streaming. Video streaming services like YouTube, NetFlix, Prime videos fall into the video-on-demand category. Here the videos are prerecorded and preprocessed. YouTube-Live, Periscope, Twitch provide live video streaming Category. The only difference between VoD and live video streaming is that videos are not preprocessed in live streaming as it is not ready yet. The services like video conferencing, webinars are fall in the third category, the interactive video. The interactive online videos are not only bidirectional but also extremely delay-sensitive. Unlike VoD or live streaming, it is okay for interactive video streaming to drop several frames than stall for data. So, the technology required for the interactive video is very different in every aspect. In our work, we concentrated on the VoD and live video streaming over HTTP only.

HTTP is a widely accepted protocol as it serves the World-Wide-Web (WWW). Most of the firewalls, proxies, and NAT-boxes allow HTTP protocol. HTTPS, the secure version of HTTP, is equally acceptable for the network administrator of different organizations. So, the HTTP(S) based video streaming services also allowed by those firewalls, proxies, and NAT boxes. This one feature favored HTTP(S) based video streaming over the existing video streaming system.

On top of the HTTP-based video streaming system, service providers can now adapt video quality according to the available network quality. It reduces the rebuffering significantly. Currently, most online video services support adaptive video streaming as it provides a better quality of experience to the viewers. Several technologies like MPEG-DASH (dynamic adaptive streaming over HTTP), Apple's HLS, and SmoothStream by Microsoft developed to provide dynamic adaptive streaming. Although different organizations develop these technologies, they work almost identically. In our work, we concentrate on the DASH only as it is a guideline than a product and the open-source implementation of DASH is available.

DASH or DASH-like video streaming systems changes video quality on the fly running a special algorithm call adaptive bitrate algorithm (ABR). The selection of ABR is crucial as the overall quality of experience is dependent on this algorithm. It is the ABR algorithms' job to provide minimal rebuffering while maintaining better video quality. In our survey, we first discuss the details of the DASH and different components of DASH. We then discuss DASH's application on YouTube, a major video streaming provider, and the latest research on the ABR to improve QoE for both VoD and live streaming in different scenarios.

