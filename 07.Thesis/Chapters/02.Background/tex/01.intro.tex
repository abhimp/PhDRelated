%\section{Introduction}
Online video streaming is the most popular service on the Internet. The pervasive penetration of smartphones and the availability of cheap LTE networks make it even easier to reach millions of users. Video streaming can be primarily categorized into three categories: i) static video or video-on-demand (VoD) streaming, ii) live video streaming, and iii) interactive video streaming. Video streaming services like YouTube, NetFlix, Amazon Prime videos fall into the video-on-demand category. Here the videos are prerecorded and preprocessed. YouTube-Live, Periscope, Twitch provide services of live video streaming category. From a player's perspective, the only difference between VoD and live video streaming is that the videos are not preprocessed in live streaming as it is not ready yet. The services like video conferencing, webinars fall into the third category, the interactive video. The interactive online videos are not only bidirectional but also extremely delay-sensitive. Unlike VoD or live streaming, it is okay for interactive video streaming to drop several frames than stall for data. So, the technologies required for interactive videos are very different in every aspect. In our work, we concentrate on the VoD and live video streaming over HTTP only.

HTTP is a widely accepted protocol as it serves the World-Wide-Web (WWW). Most of the firewalls, proxies, and NAT-boxes allow HTTP protocol. HTTPS, the secure version of HTTP, is equally acceptable for the network administrator of different organizations. So, the HTTP(S) based video streaming services also allowed by those firewalls, proxies, and NAT boxes. This particular feature increases the popularity of HTTP(S)-based video streaming over the existing video streaming systems.

On top of the HTTP-based video streaming system, service providers can adapt video quality according to the available network quality using technology like MPEG-DASH (Dynamic Adaptive Streaming over HTTP), Apple's HTTP Live Streaming (HLS), or Microsoft's SmoothStreaming. These technologies reduce the rebuffering significantly. Currently, most online video services support adaptive video streaming as it provides a better quality of experience to the viewers. Although different organizations develop these technologies, they work almost identically. In our work, we concentrate on DASH-based video streaming as it is a guideline than a product, which is widely used in today's video streaming services. Furthermore, the open-source implementation of DASH is available for wide-scale implementation and testing.

Depending on the video-streaming environment like the available network bandwidth, DASH or DASH-like video streaming systems adapt the video quality on-the-fly by running a special algorithm called \textit{Adaptive Bitrate} (ABR) algorithm. The selection of ABR is crucial as the overall quality of experience is dependent on the underlying ABR algorithm. It is the ABR algorithms' job to provide minimal rebuffering while maintaining better video quality. In this chapter, we discuss the basic principles of DASH and the details of different components of DASH.  We then discuss the latest researches on the ABR to improve QoE for both VoD and live streaming in different scenarios and finally conclude with the available research scopes.

