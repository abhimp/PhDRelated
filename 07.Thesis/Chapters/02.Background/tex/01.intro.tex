\section{Introduction}
Online video streaming is the most popular service on the Internet. The pervasive penetration of smartphone and availability of cheap LTE network make it even easier for reach to millions of users.
%The introduction of the video streaming system over the HTTP and in build player in HTML5 makes it easier to start a new video streaming service on the one hand. On the other hand, the LTE network's pervasive penetration makes it easier for smartphone users to watch online videos. On top of HTTP and HTML5, dynamic adaptation of the audio-video quality allows stream providers to make the service tolerant to frequent network change. These technologies gain the attention of researchers all around the globe. In this work, we discuss the technological advancement for video streaming using HTTP.
Video streaming can be primarily categorized into three categories: i) static video or video-on-demand (VoD) streaming, ii) live video streaming, and iii) interactive video streaming. Video streaming services like YouTube, NetFlix, Prime videos fall into the video-on-demand category. Here the videos are prerecorded and preprocessed. YouTube-Live, Periscope, Twitch provide live video streaming Category. The only difference between VoD and live video streaming is that videos are not preprocessed in live streaming as it is not ready yet. The services like video conferencing, webinars are fall in the third category, the interactive video. The interactive online videos are not only bidirectional but also extremely delay-sensitive. Unlike VoD or live streaming, it is okay for interactive video streaming to drop several frames than stall for data. So, the technology required for the interactive video is very different in every aspect. In our work, we concentrated on the VoD and live video streaming over HTTP only.

HTTP is a widely accepted protocol as it serves the World-Wide-Web (WWW). Most of the firewalls, proxies, and NAT-boxes allow HTTP protocol. HTTPS, the secure version of HTTP, is equally acceptable for the network administrator of different organizations. So, the HTTP(S) based video streaming services also allowed by those firewalls, proxies, and NAT boxes. This one feature favored HTTP(S) based video streaming over the existing video streaming system.

On top of the HTTP-based video streaming system, service providers can now adapt video quality according to the available network quality. It reduces the rebuffering significantly. Currently, most online video services support adaptive video streaming as it provides a better quality of experience to the viewers. Several technologies like MPEG-DASH (dynamic adaptive streaming over HTTP), Apple's HLS, and SmoothStream by Microsoft developed to provide dynamic adaptive streaming. Although different organizations develop these technologies, they work almost identically. In our work, we concentrate on the DASH only as it is a guideline than a product and the open-source implementation of DASH is available.

DASH or DASH-like video streaming systems changes video quality on the fly running a special algorithm call adaptive bitrate algorithm (ABR). The selection of ABR is crucial as the overall quality of experience is dependent on this algorithm. It is the ABR algorithms' job to provide minimal rebuffering while maintaining better video quality. In our survey, we first discuss the details of the DASH and different components of DASH. We then discuss DASH's application on YouTube, a major video streaming provider, and the latest research on the ABR to provide better QoE for both VoD and live streaming in different scenarios.

%\subsection{DASH}
%
%%
%Dynamic Adaptive Streaming over HTTP (DASH), also referred to as MPEG-DASH, is an adaptive bit-rate solution for video streaming, which enables client-operated video delivery over HTTP.
%%
%DASH is implemented by breaking down the video content into small segments, each worth a short duration of playback time.
%%
%For every segment of playback time, alternative versions at various bit-rates are available at the server.
%%
%The client typically requests for the highest quality segment possible under current network conditions, such that it is received (downloaded) in time for playback, without causing stalling or re-buffering. 
%%
%However, DASH is not a protocol -- it only specifies an architecture (Fig.~\ref{fig:dash}) to enable adaptive video streaming over HTTP.
%%
%Every video streaming service (e.g., YouTube, Netflix, etc.) is free to define its own implementation of the DASH modules.
%
%\subsection{ABR}
%Adaptive bitrate algorithm or ABR algorithm is the heart of the DASH. ABR decide when and which quality to download based on network quality and other parameter. Primary job of any ABR algorithm is improve user experience by selecting appropriate video quality for the a segment. As the online video streaming become one of the most popular service in the Internet, it become more prevalent to improve ABR algorithm to provide better QoE in different. The problem attract researcher from academic as well as the industries. Researcher start exploiting different areas of video streaming with a common goal, i.e. improving QoE for end user.
%
%\subsection{QoE}
%Quality of Experience is a important parameter in any field which involves end users. In case of the video streaming, QoE is metric to measure whether user have enjoyed the video or not. QoE is mostly a user perspective and it depends on several factor such as startup delay, quality flactuation, overall quality, rebuffering, audio/video device, video content. Although all this parameters are important, it is difficult to measure user dependent parameter and video content. Several researcher tried to to find best measurable metric to calculate QoE which can be acceptable. Mok \etal \cite{5990550} tried to calculate QoE from the network QoS. Infact they suggested that user QoS (or QoS of HTTP) is the QoE. 
%
%\blue{TODO: More about QoE}
