\section{Video Streaming over Smartphones}
The availability of cheap data plan over the LTE network and the availability of contents in the regional languages led to an increase in the video playback time in smartphones to a magnitude. There is no doubt that researchers and industries have started investing their resources to improve the smartphone's video streaming experience. The smartphone comes with a variety of screen sizes and battery capacity. These variations make it even challenging to provide the best possible video quality while draining the least amount of battery backup, so that viewers can enjoy the video to the fullest for a longer time. There are three components -- (a) screen, (b) decoding hardware/software, and (c) radio in a smartphone, which consume energy during video streaming. A carefully designed video streaming system can reduce the energy consumption in all the three components. For example, a decrease in the startup delay and rebuffering directly decreases the unnecessary screen on time, which decreases the energy requirement by the online video playback. An efficient codec can reduce energy consumption by reducing decoding overhead and intelligent use of the cellular radio to reduce smartphones' energy requirement during video streaming. Among these parameters, the codec is mostly static throughout the video playback session. It is easy to decide the suitable codec based on the smartphone variant at the beginning of the playback session. However, reducing rebuffering and optimizing radio usage is challenging due to various parameters, including the device's mobility pattern, cell-tower distribution, cell-tower load, and so on. This section discusses a few of the various techniques developed specifically for online adaptive bitrate video streaming over the smartphones.

\subsection{Energy-aware Video Streaming}
Video streaming is an extremely power-hungry service. Prefetching using a WiFi network can reduce power consumption \cite{6681586,10.1145/2079296.2079321}. However, it causes a lot of data wastage as it is impossible to predict users' video preference for an extended period accurately. Hu \etal\ have proposed a solution to make video streaming energy efficient by ON-OFF scheme \cite{7218493}. The scheme exploits the energy states of LTE radio, which is consist of 3 states -- (a) Active/On (high energy), (b) Tail (medium energy), and (c) Off (low energy). The jump from Off to On state requires promotion energy. No jump from Off state to Tail state or On state to Off state is possible. Hu \etal\ have suggested that a smartphone has a fixed buffer, and when the radio state is in the On state, the app should fill the buffer before it goes to Tail state. The primary goal here is to minimize the tail time without involving bitrate adaptation.

\subsection{Energy Consumption by DASH over LTE}
Zhang \etal\ \cite{10.1145/2910018.2910656} have measured the power consumption by DASH video streaming over LTE. They measured the energy consumption using Monsoon power monitor \cite{monsoonmonitor} tool on various streaming strategies and conditions. The study yields that network-based energy consumption can be reduced up to 30\% just by changing the segment length from 2 sec to 4 sec. Similarly, an increase in the buffer size can also reduce the energy consumption.

\subsection{OSCAR}
OSCAR~\cite{10.1145/2910018.2910655} is a hybrid ABR algorithm specially designed for the smartphone to reduce the stall while in mobility. Zahran \etal\ have modeled the throughput as a random variable with Kumaraswamy distribution \cite{jones2009kumaraswamy} to estimate the rebuffering probability. They used an adaptation technique similar to the buffer-based ABR algorithms with three thresholds low, transient, and high. At low buffer condition, it just downloads the lowest quality segment to fill the buffer. However, the algorithm tries to avoid radio off state by downloading the high (maximum among highest supported by the estimated throughput and 1-level up from the last quality) quality segment. A segment's quality is determined by solving an optimization problem to maximize the sum of an exponential video utility function and switching penalty. As per the simulation result, the solution yields up to 85\% stall-free playback time.

\subsection{Summary}
Although DASH attracts researchers to study, design, and develop ABR algorithms and streaming solutions to improve QoE, however, very few of them are concerned about energy consumption by online video streaming over a smartphone. In this section we have discussed a couple of such researches. While one of them has studied the energy consumption~\cite{10.1145/2910018.2910656}, the other provides an ABR algorithm to improve the power efficiency~\cite{10.1145/2910018.2910655}. However, to the best of our knowledge, there is no solution to provide an energy-efficient video stream solution while the user is mobile.