\section{Video streaming to SmartPhone}
With the availability of cheap data plan over LTE network and content availability in local language led to increase in video playback time in smartphone to a record. No doubt that researchers and industries start investing their resources to improve the video streaming experience over smartphone. SmartPhone comes with variety of screen sizes and battery capacity. These variation become goal to provide best possible video quality while draining least amount of battery backup so that viewer can enjoy video to the fullest for a longer time. There are three component consume energy during video streaming over a smartphone. These are a) screen, b) decoding hardware and c) radio. There are scope to reduce energy consumption in all three component. Rebuffering increase the screen on time, thus reducing rebuffering reduces the energy requirement by the online video playback. Use of efficient codec can reduce the energy consumption by the decoding hardware and intelligent use of the cellular radio can reduce energy requirement by the same. Among these parameter, codec  mostly static and easy to decide the suitable codec based on the smartphone variant. However, reducing rebuffering and optimizing radio is challenging due to various parameters including device's mobility pattern, cell-tower distribution, cell-towerload etc. In this section we discus few of the various technique developed specifically for online video streaming over smartphone.

\subsection{Energy-Aware video streaming}
Video streaming is extremely power hungry service. Prefetching can be used to reduce the power consumption \cite{6681586,10.1145/2079296.2079321}, however it cause lot of data wastage which is not cheap that time. Hu \etal\ proposed a solution to make video streaming energy efficient by On-Off scheme \cite{7218493}. The scheme exploit the energy states of LTE radio which is consist of 3 state a) Active/On (high energy), b) Tail (medium energy) and c) Off (low energy). The jump from Off to On state require promotion energy. No jump from Off to Tail or On to Off is possible. Hu \etal\ suggested that smartphone should have fixed buffer and when the radio state is in On state, app should fill the buffer before it goes to Tail state.

\subsection{Energy consumption by DASH over LTE}
Zhang \etal\ \cite{10.1145/2910018.2910656} measured power consumption by DASH video streaming over LTE. They measured the energy consumption using Monsoon power monitor \cite{monsoonmonitor} tool on various streaming strategies and conditions. The study yelds that network based energy consumption can be reduced upto 30\% just by change segment length from 2 sec to 4 sec. Similarly increase buffer size can also reduce energy consumption.

\subsection{OSCAR}
OSCAR\cite{10.1145/2910018.2910655} is hybrid ABR algorithm specially designed for smartphone to reduce the stall while in mobility. Zahran \etal\ model the throughput as a random variable with Kumaraswamy distribution \cite{jones2009kumaraswamy} to estimate the stall probability. The adaptation technique is similar to the buffer-based ABR algorithms with three thresholds low, transient and high. At low buffer condition, it just download the lowest quality segment to fill the buffer. However the algorithm tries to avoid radio off state by downloading the high (maximum among highest supported by the estimated throughput and 1level up from the last quality) quality segment. The quality of a segment is determined by solving a optimizing problem with goal of maximizing a sum of exponential video utility function and switching penalty. As per the simulation result the solution yields upto 85\% stall free playback time.

\subsection{Summary}
Although DASH attract researcher to study, design and develop ABR algorithm, streaming solution to improve QoE. However very few of them concerned about the energy consumption by online video streaming on smartphone. We discuss few such research in this section. Some of these researches, studied the energy consumption\cite{10.1145/2910018.2910656} and few provides algorithm to improve power efficiency\cite{10.1145/2910018.2910655}. However, there is no solution to provide energy efficient video stream solution while the user is mobile.