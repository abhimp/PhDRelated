\section{Classical ABR Algorithms}
There are many classical ABR algorithms designed from the inception of DASH-like video streaming. Any classical ABR algorithm's primary goal is to run on a low-powered device based on the passive measurement of network bandwidth. The classical ABR algorithms can be further categorized in {\tt buffer-based}, {\tt throughput-based}, and {\tt hybrid} ABRs. We discuss the classical ABR algorithms from each category in this section.

\subsection{Throughput-based ABR algorithm}
Any ABR algorithm's primary job is to play the video without rebuffering. To achieve this, historically, ABR algorithms used to start with the lowest bitrate to quickly start the video and improve the quality in subsequent segments until it detects some congestion or loss~\cite{5677508,10.1145/1943552.1943575}. While this technique avoids rebuffering, it keeps on changing the video quality for almost all the segment. Microsoft started using a little more conservative solution in~\cite{10.1145/1943552.1943574} while adapting the bitrate based on the network quality. These algorithms have mitigated the rebuffering by matching the video bitrate with the player's instantaneous network throughput. However, they have failed to consider that frequent changes in the quality can be an issue. These algorithms have the goal to improve the average quality and reduce the rebuffering. Later on, the ABR algorithms are designed to maintain the QoE. Few such algorithms are discussed below. 

\subsubsection{QoE aware DASH (QDASH)}
Mok \etal\ designed the QDASH~\cite{10.1145/2155555.2155558}, one of the early ABR algorithms, with the awareness of QoE, considering that the abrupt quality changes are not acceptable. The algorithm has two parts, a) ABR algorithm and b) Bandwidth measurement tool. They proposed a particular module to run at the streaming server, which measures the client bandwidth. The QDASH ABR algorithm explicitly connects the measurement module and get the bandwidth information. With the bandwidth information, QDASH finds the most suitable bitrate. However, QDASH does not switch to the most appropriate bitrate immediately; instead, it switches to the next bitrate to increase the QoE.

\subsubsection{Fair, Efficient, and Stable adapTIVE (FESTIVE)}
FESTIVE~\cite{10.1145/2413176.2413189} is a DASH-based video streaming system that provides fairness between players that share a bottleneck. Jiang \etal\ showed that the measured throughput might not be correct, and the available bottleneck bandwidth might be underutilized due to the scheduling scheme. So, they suggested a three steps method. These steps are --
(a) schedule the next segment download in a randomized manner so that every player gets a fair share of bottleneck links,
b) select suitable bitrate based on the last bandwidth estimation, but never jump to that bitrate, instead increase one level at a time,
c) estimate the available bandwidth as the harmonic mean of the last 20 throughput measurements.


\subsubsection{FEedback Linearization Adaptive STreamIng Controller (ELASTIC)}
All the client-side adaptive algorithms generate an ON-OFF traffic pattern, which led to unfairness among the players. Cicco \etal\ designed ELASTIC~\cite{6691442}, a client-side controller that does not generate ON-OFF traffic patterns. ELASTIC selects video bitrate for a segment in such a way so that it finishes downloading the segment at the time when the next segment needs to be downloaded. So, there is no requirement for the ON-OFF traffic generation, and traffic can be fair as the underlying TCP is fair for the long-running flows.

\subsubsection{Presto: Fair and Efficient Streaming using Multiple Servers}
Zhang \etal\ designed Presto~\cite{7249417}, which provides fairness among the players that stream video from multiple servers. They argue that a player might consume more bandwidth of the bottleneck by running more parallel flows. So, they provide a mechanism to restrict a player's bitrate with more flows and improve the quality. Presto exploits that a user gets annoyed with drastic bitrate drop; however, not with drastic bitrate increase~\cite{10.1145/2018602.2018611}. It also suspends some flows to stabilize the bandwidth utilization for the time and resumes those flows later.

\subsubsection{Spectrum-based QUality ADaptation (SQUAD)}
SQUAD~\cite{10.1145/2910017.2910593} is a spectrum-based~\cite{1386243} quality adaptation technique for DASH-based video streaming system. Wang \etal\ assumed spectrum as the variation in the bitrate around some $N$ number of segments. The proposed system starts with providing a new method to measure the throughput. Typically throughput is measured when a segment is completely downloaded, and estimations are made based on the past throughput measurement. The authors have pointed out that this technique may not be correct as the sizes of the segments are different, and thus it might take different times to download different segments even if the bandwidth is the same. So, SQUAD suggested performing a running average of throughput throughout a segment downloading. It proposed to set the bitrate of a few initial segments to the lowest bitrate available and, after the initial set of segments, set bitrate in such a way so that the bitrate variation in a window, i.e., the spectrum, is the lowest.


\subsection{Buffer-based ABR Algorithms}
The buffer-based bitrate adaptation technique is an alternative technique of throughput-based, where the player does not need to estimate or measure the throughput to adapt the playback quality, as the throughput estimation is complex and most of the time inaccurate due to the nature of network dynamics. Also, middleboxes like catching devices, including proxy, make it more difficult to estimate real throughput. Buffer-based ABR algorithms do not suffer from a similar problem. In the past decade, researchers have designed several buffer-based ABR algorithms. A few buffer-based ABR algorithms are discussed next. 

\subsubsection{Buffer-based Rate Adaption}
Huang \etal~\cite{Huang2014,10.1145/2398776.2398800,10.1145/2491172.2491179} described a bitrate adaptation mechanism solely based on the playback buffer level. The authors suggested that the playback buffer is a function of bitrate and network throughput. Thus the player buffer level can be an indication of whether to change the bitrate or not. The authors have used a \textit{Rate to Buffer Map} to calculate the bitrate for a given buffer. They used an algorithm to remove any fine boundary between bitrates to avoid frequent bitrate fluctuations.

\subsubsection{Threshold Driven Buffer-based Adaptation}
Miller \etal\ designed a buffer based algorithm in \cite{6229732} as a remedy for the measurement problem of throughput driven algorithms. They uses 3 buffer levels, $B_{min}$, $B_{low}$ and $B_{max}$ ($0 < B_{min} < B_{low} < B_{max}$) as the thresholds to decide the bitrate switch. The algorithm aims to keep the buffer level close to the mean of $B_{low}$ and $B_{max}$. It decreases the bitrate if the buffer level goes below $B_{low}$, drops to the lowest if it goes below $B_{min}$, and increases if it goes beyond $B_{max}$.

\subsubsection{Smooth Adaptive Bit RatE (SABRE)}
SABRE is designed to mitigate TCP's buffer bloat effect, which might cause queuing delay up to a few seconds. Mansy \etal~\cite{10.1145/2483977.2484004} suggested using the HTTP pipeline where multiple HTTP-requests can be pushed together. At the client-side, they proposed to make the {\tt recv} call in a non-aggressive way to reduce the queuing delay as it prevents the sender side from pushing excessive data. Due to the paced {\tt recv} call, it is impossible to correctly measure the throughput. It changes the pacing of the {\tt recv} call based on the playback buffer. The player changes the bitrate as per the observed download rate, which is governed by the non-aggressive pacing.

\subsubsection{Buffer Occupancy-based Lyapunov Algorithm (BOLA)}
BOLA~\cite{Spiteri2016,bola2-acm-mmsys2018} treats ABR streaming as an optimization problem where the average bitrate needs to be maximized, and the rebuffering needs to be minimized. Spiteri \etal\ solved the problem using Lyapunov optimization and provided a theoretical guarantee on the QoE. BOLA does not require any estimation of the throughput. Instead, it is solely based on the playout buffer.

\subsubsection{Adaptation \& Buffer Management Algorithm (ABMA+)}
Beben \etal\ proposed ABMA+~\cite{10.1145/2910017.2910596} which predicts the video freezing probability for each representation and selects the best representation that satisfies the predefined freezing probability threshold. It continuously monitors the segment download time and uses time characteristics, and the precomputed buffer-map decides the best bitrate for smooth playback.

\subsubsection{Markov Decision-Based Rate Adaptation Approach (mDASH)}
mDASH~\cite{7393865} formulates the rate adaptation problem as a Markov Decision Process (MDP)~\cite{P-1066} optimization. According to Zhou \etal\, the state vector defines the system state, including buffer occupancy, playback rate, previous video bitrate, and network bandwidth. The action is the bitrate for the next segment. They calculate the reward based on various parameters that affect the visual quality, including quality, quality switch frequency and magnitude, and rebuffering. The optimization problem is difficult to solve directly as there are many uncertainties during the streaming session due to time-varying network conditions, so the authors proposed a sub-optimal greedy solution to solve the problem. According to their evaluation, mDASH performs at per with the optimal solution.

\subsection{Hybrid ABR Algorithms}
Some ABR algorithms consider both buffer and throughput to adapt the bitrate. These algorithms can be further categorized as control system-based and without control system-based. We discuss a few of them here.

\subsubsection{Smooth Rate Adaptation}
The authors of \cite{10.1145/2413176.2413190} and \cite{6694183} proposed a method involving a control loop and measured the throughput and the playback buffer. The control loops start with a playback buffer and a reference buffer, and the difference between these two is used to predict the throughput. Then the predicted throughput is used to determine the video bitrate. The video bitrate and the real throughput lead to playback buffer occupancy, further used in the control loop.

\subsubsection{Probe and Adapt (Panda)}
Li \etal~\cite{140405} found that most of the widely deployed DASH like streaming systems suffer from bitrate fluctuation when multiple players share a bottleneck. They also found that the fluctuation depends on various parameters, including the number of players, the players' start time, and the background traffic. While digging more, they found that all the techniques assume that the measured throughput is fair and can be used directly. However, it is not valid, and due to this fundamental mistake, a video player loses the capability of overcoming a buffer underrun. Li \etal\ suggested using probing over the network to find the available bandwidth. PANDA uses TCP like Additive Increase Multiplicative Decrease (AIMD) scheme for the rate-adaptation. However, it only uses the probe per chunk instead of per round-trip time (RTT).

\subsubsection{Segment Aware Rate Adaptation (SARA)}
SARA~\cite{7247436} is a segment aware rate adaption technique developed by Juluri \etal. It is a hybrid adaptive system where the playback buffer is used to decide how the bitrate should change, and the throughput is used to decide the appropriate bitrate. It has bitrate adaptation scheme based on four buffer thresholds. These are (a) \textit{Slow-start}: when buffer ($B_{curr}$) is very low ($B_{curr}<I$), lowest bitrate is selected, (b) \textit{Additive increase}: when the buffer is moderate ($B_{\alpha} \le B_{curr} > B_{\beta}$), it carefully increase the bitrate, (c) \textit{Aggressive bitrate switching}: if the buffer is high enough ($B_{\beta} \le B_{curr} \ge B_{max}$, bitrate can jump without any effect, and (d) \textit{Delayed download}: if the buffer is saturated ($B_{max} < B_{curr}$), the download has to wait for the buffer to go down.

\subsubsection{Model Predictive Control Algorithm}
Model predictive control (MPC) algorithm is proposed by Yin \etal\ in their paper~\cite{yin2015control,10.1145/2670518.2673877} as a superior ABR algorithm than the existing algorithms by optimally combining the throughput and buffer occupancy information. The authors formulated the QoE optimization problem as a stochastic optimal control problem and solved it using MPC. They develop the bitrate selection as a function of three components -- (a) predicted future bitrate, (b) buffer occupancy, and (c) available bitrates. The basic MPC algorithm is made up of three steps: \\
(i) \textit{Predict}: Although the future throughput prediction is difficult and most of the time inaccurate, it assumes that there is a way to do it with acceptable errors. In the throughput prediction step, they can use any third party algorithm to predict the throughput.\\
(ii) \textit{Optimize}: In this phase, the MPC search for the bitrate for the next $N$ segments with predicted throughput and calculated buffer occupancy, and choose the best one. The problem can be solved using {\tt CPLEX} solver.\\
(ii) \textit{Apply}: Once bitrate is selected, it should start downloading the segment with the desired bitrate.\\
The optimization step is computationally heavy, and it has to perform before downloading each segment. So, the authors have proposed an alternative to making it fast by using lightweight combinations.


\subsubsection{Fine-Grained Video Rate Adaptation (Favor)}
Fine-grained Video Rate Adaptation or Favor~\cite{10.1145/3204949.3204957} has been proposed by He \etal\ to optimize the non-conventional parameters like frame-dropping rate and playback speed along with convention parameter bitrate to optimize DASH-based players beyond the optimization horizon. The author suggested that the viewer cannot notice up to 35\% frames drops and up to 25\% reduction in the playback speed. Although these changes are non-conventional, it can allow the player to cope with the throughput reduction as frame reduction causes a decrease in segment size, and playback rate reduction gives more time to download a segment. Favor also suggest a framework for 360$^{\circ}$ video by tiling and optimizing individual tiles.

\subsubsection{DASH with Peer-to-Peer}
Yousef \etal\ designed a technique in their paper~\cite{10.1145/3339825.3391859} to allow any ABR algorithm to work in a hybrid CDN+P2P DASH based video streaming system. They suggested keeping the video player, and the P2P modules separate so that any DASH-based ABR algorithm can be applied to the player. The authors have shown that the main problem with P2P DASH-based streaming is the estimation of the throughput or the buffer filling rate, as the throughput and delay vary drastically depending on whether the video is being downloaded from a CDN or a nearby peer. It maintains the player's observed throughput steady by adding extra delay in their P2P module while downloading segments from a nearby peer. This way, any DASH-based ABR algorithm can work in the player.

\subsection{Summary}
In summary, there are several classical ABRs algorithms developed over the years. It starts with simple throughput driven ABR algorithms, where ABR directly matches the streaming quality to the last observed network throughput~\cite{5677508,10.1145/1943552.1943575,10.1145/1943552.1943574}. Although it reduces the rebuffering, quality change is imminent. So researchers tried to tune the throughput measurement and quality switching policy to improve the QoE. However, throughput-based algorithms require continuous throughput measurement, which is impossible due to the ON-OFF traffic pattern. The fine-grain throughput measurement is not possible from the application layer due to the network stack's nature. So, researchers started developing buffer-based ABR algorithm as the buffer can be an indication of throughput. Buffer-based algorithms adapt the bitrate with or without having a bitrate map. However, the frequent quality switch is still a problem. So both throughput-based and buffer-based algorithms applied mechanisms to avoid strict changes in the quality.

While throughput and buffer-based ABR algorithms provide reasonably better QoE, there are places to improve. It cannot always provide the best solution due to the nature of the network dynamics -- a Hybrid solution can rescue ABRs from those pitfalls. ABR algorithms in \cite{7247436,140405,yin2015control,10.1145/2670518.2673877} perform much better than any buffer-based or throughput-based algorithms. These algorithms assume the QoE maximization as an optimization problem and try to solve it. However, there are problems in deploying hybrid ABR algorithms as these are complex, computationally heavy, and require a special solver to solve the problem. Although algorithm designers usually provide a lightweight solution, those are not adequate.

All the existing classical ABR algorithms have some advantages as well as disadvantages. We compare those algorithms in terms of a few basic parameters in \tbl{\ref{chap02:tbl:comparison_classical}}.

% Please add the following required packages to your document preamble:
% \usepackage{multirow}
\begin{table}[h!]	
	\caption{\label{chap02:tbl:comparison_classical}Comparisons of classical ABR Algorithms}
	\resizebox{\textwidth}{!}{
		\begin{tabular}{|l|cccc|c|c|l|}
			\hline
			\multicolumn{1}{|c|}{\multirow{3}{*}{Algorithm}} & \multicolumn{4}{c|}{Goal involved} & \multirow{3}{*}{\begin{turn}{90}No Extra Module\end{turn}} & \multirow{3}{*}{\begin{turn}{90}Continuous Traffic\end{turn}} & \multicolumn{1}{c|}{\multirow{3}{*}{Remark}} \\
			\cline{2-5}
			& \begin{turn}{90}Rebuffering\end{turn} & \begin{turn}{90}Quality\end{turn} & \begin{turn}{90}Smoothness\end{turn} & \begin{turn}{90}Fairness\end{turn} & & & \\
			& & & & & & & \\
			\hline
			\multicolumn{8}{|c|}{Throughput Based}\\
			\hline
			QDASH\cite{10.1145/2155555.2155558} & \green{\cmark} & \green{\cmark} & \green{\cmark} & \red{\xmark} & \red{\xmark} & \red{\xmark} & Match observed bitrate \\
			FESTIVE\cite{10.1145/2413176.2413189} & \green{\cmark} & \green{\cmark} & \green{\cmark} & \green{\cmark} & \green{\cmark} & \red{\xmark} & \begin{tabular}[c]{@{}l@{}}Harmonic Mean\\ No bitrate jump\end{tabular} \\
			ELASTIC\cite{6691442} & \green{\cmark} & \green{\cmark} & \red{\xmark} & \green{\cmark} & \green{\cmark} & \green{\cmark} & Fairness from long running TCP \\
			Presto\cite{7249417} & \green{\cmark} & \green{\cmark} & \green{\cmark} & \green{\cmark} & \green{\cmark} & \red{\xmark} & Multi-flow down stream per player \\
			SQUAD\cite{10.1145/2910017.2910593} & \green{\cmark} & \green{\cmark} & \green{\cmark} & \red{\xmark} & \green{\cmark} & \red{\xmark} & \begin{tabular}[c]{@{}l@{}}Specturm based\\ Running average throughput\end{tabular} \\
			\hline
			\multicolumn{8}{|c|}{Buffer Based}\\
			\hline
			Buffer based rate Adaptation\cite{Huang2014,10.1145/2398776.2398800,10.1145/2491172.2491179} & \green{\cmark} & \green{\cmark} & \red{\xmark} & \red{\xmark} & \green{\cmark} & \red{\xmark} & rate2buffer map \\
			Threshold Based\cite{6229732} & \green{\cmark} & \green{\cmark} & \red{\xmark} & \red{\xmark} & \green{\cmark} & \red{\xmark} & Buffer threshold decides bitrate \\
			SABRE\cite{10.1145/2483977.2484004} & \green{\cmark} & \green{\cmark} & \red{\xmark} & \red{\xmark} & \green{\cmark} & \red{\xmark} & Avoids buffer bloat \\
			BOLA\cite{Spiteri2016} & \green{\cmark} & \green{\cmark} & \red{\xmark} & \red{\xmark} & \green{\cmark} & \red{\xmark} & Theoretical Guarantee on video quality \\
			ABMA+\cite{10.1145/2910017.2910596} & \green{\cmark} & \green{\cmark} & \green{\cmark} & \red{\xmark} & \green{\cmark} & \red{\xmark} & Buffer freezing probability threshold \\
			mDASH\cite{7393865} & \green{\cmark} & \green{\cmark} & \green{\cmark} & \red{\xmark} & \green{\cmark} & \red{\xmark} & MDP optimization problem \\
			\hline
			\multicolumn{8}{|c|}{Hybrid} \\
			\hline
			Smooth rate adaptation\cite{10.1145/2413176.2413190,6694183} & \green{\cmark} & \green{\cmark} & \red{\xmark} & \red{\xmark} & \green{\cmark} & \red{\xmark} & \begin{tabular}[c]{@{}l@{}}Control loop\\ Throughput prediction\end{tabular} \\
			PANDA\cite{140405} & \green{\cmark} & \green{\cmark} & \green{\cmark} & \green{\cmark} & \green{\cmark} & \red{\xmark} & \begin{tabular}[c]{@{}l@{}}Probe network by increased datarate\\ AIMD bitrate selection\end{tabular} \\
			SARA\cite{7247436} & \green{\cmark} & \green{\cmark} & \red{\xmark} & \red{\xmark} & \green{\cmark} & \red{\xmark} & \begin{tabular}[c]{@{}l@{}}Buffer used to decide selection algo\\ Match measured throughput\end{tabular} \\
			MPC\cite{yin2015control,10.1145/2670518.2673877} & \green{\cmark} & \green{\cmark} & \green{\cmark} & \red{\xmark} & \green{\cmark} & \red{\xmark} & Assume throughput is predictable \\
			FAVOR\cite{10.1145/3204949.3204957} & \green{\cmark} & \green{\cmark} & \green{\cmark} & \red{\xmark} & \red{\xmark} & \red{\xmark} & \begin{tabular}[c]{@{}l@{}}Changes playback speed\\ Drop frames\end{tabular}\\
			\hline
		\end{tabular}
	}
\end{table}
