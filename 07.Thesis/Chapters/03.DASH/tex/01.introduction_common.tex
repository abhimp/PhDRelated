\blue{Online video streaming service is one of the most popular online services on the Internet. As per Sandvine's\footnote{\url{https://www.sandvine.com/hubfs/downloads/phenomena/2018-phenomena-report.pdf}} report, video streaming takes almost 58\% of total down-stream traffic of the global Internet traffic. Google's YouTube, already a part of the common Internet parlance, has emerged as the largest player in the mobile video market, accounting for 40--70\% of total video traffic across most mobile networks\footnote{\url{https://www.ericsson.com/assets/local/mobility-report/documents/2016/ericsson-mobility-report-november-2016.pdf}}.
Dynamic Adaptive Streaming over HTTP(S) or DASH is a technology to streaming video over HTTP/HTTPS protocols. All the video streaming services including YouTube, NetFlix usages technology similar to DASH. The DASH is resilient to firewall and NAT filters as traffic goes through widely accepted HTTP and HTTPS protocol.
}

\blue{
The DASH and DASH like streaming system are hot topic for the researcher. Google developed a new transport protocol QUIC along with other optimization in the YouTube player to improve the user experience. In this chapter we perform analysis on video streaming system of YouTube, effect of QUIC protocol on DASH based streaming and YouTube's streaming behavior while the user is mobile.
}

%\blue{YouTube being one of the biggest player in the video streaming service, Google developed various optimizations and technologies to provide better playback experience to its users.
%Not surprisingly, YouTube has garnered significant interest in the research community over the years, furnishing studies which explore various aspects of the service -- a large majority of which focus on its video playback mechanism.
%However, the interest in YouTube's video streaming behavior is far from satiated -- a phenomenon largely propelled by YouTube's practice of incessant technical evolution.
%}