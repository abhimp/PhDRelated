%\subsection{Introduction}
%\blue{
%In last section we performed analysis on the YouTube's video streaming system. Google have developed a new transport protocol 
%}


%Quick UDP Internet Connection (QUIC)~\cite{langley2017quic} has been developed and experimentally deployed over the Internet by Google to replace TCP as the transport layer protocol, while addressing the limitations of TCP for end-to-end connection managements over a wide range of applications. While majority of the global Internet traffic originates from various video streaming applications, QUIC claims to reduce the YouTube rebuffering by a margin of 18\% for desktop users and 15.3\% for mobile users~\cite{langley2017quic}. As Dynamic Adaptive Streaming over HTTP (DASH)~\cite{stockhammer2011dynamic} has been a de facto for Adaptive Bitrate Streaming (ABR) over the Internet, it would be interesting to explore how QUIC performs over various ABR techniques proposed in the recent literature in terms of end users' quality of experience (QoE).


%In DASH, the videos are divided into small segments, and every segment is encoded in multiple quality levels (bitrates). The DASH client measures the network condition and requests for a video segment with the most suited quality level based on the current network condition. The network measurement and the corresponding bitrate adaptation algorithm can be tuned in DASH, and a large number of works have been proposed in the literature to select the optimal ABR technique, such as buffer based (BOLA~\cite{Spiteri2016}), using control-theoretic approach (MPC~\cite{yin2015control}), or based on deep reinforcement learning (Pensieve~\cite{mao2017neural}). However, these existing approaches do not look into the impact of the underlying transport protocols on the video streaming QoE performance. While TCP uses multiple socket connections to download the video and the corresponding audio data, QUIC multiplexes the audio and the video streams over a single UDP socket to download the entire data from the server. 
%As the ABR techniques depend on the channel throughput estimation at the client, such protocol changes are likely to impact the streaming performance. 


%With the above context, this work evaluates and compares the performance of various ABR techniques over TCP and QUIC. We develop a testbed setup with the help of a standard DASH player from the DASH Industry Forum, where we integrate the DASH player with QUIC and use emulated network environment based on a large pool of pre-collected network traffic traces.
%With a total of $45$ hours of streaming video data, we have computed three QoE metrics -- (a) average playback bitrate, (b) total rebuffering duration, and (c) playback smoothness, while streaming over both TCP and QUIC. Our thorough experiments and observations indicate that recent ABR techniques provides better QoE over TCP compared to QUIC. We investigate further to understand the protocol-level behavior of QUIC, which impacts the QoE performance.
%Our analysis reported in this work can help the community to tweak the QUIC configurations to obtain the best QoE performance from the modern ABR techniques.
The video streaming service works at the application layer of the TCP/IP networking stack, so do the ABR algorithms. Thus these services can observe the application layer throughput only. However, application layer throughput also depends on the underlying transport protocol. For example, TCP itself has several congestion control algorithms for different scenarios. Further, there are two different transport protocols, TCP and QUIC, that are widely used nowadays for supporting video streaming applications. All these things can affect the decision of an ABR algorithm. 

To understand how QUIC works with the state-of-the-art adaptive streaming protocols, we perform a study to compare the performance of ABR algorithms on top of QUIC and TCP. To do this study, we use open source DASHIF player as it is openly available, and we can set the testbed over any platform of our choice. The detailed experimental setup follows. 

As we discussed before, any ABR algorithms' performance depends on various factors, including the underlying transport protocol. 
Recently Google has developed and experimentally deployed over the Internet a new transport protocol called \textit{Quick UDP Internet Connection} (QUIC)~\cite{langley2017quic} to replace TCP as the transport protocol, while addressing the limitations of TCP for end-to-end connection managements over a wide range of applications. While the majority of the global Internet traffic originates from various video streaming applications, QUIC claims to reduce the YouTube rebuffering by a margin of 18\% for desktop users and 15.3\% for mobile users~\cite{langley2017quic}.  It is interesting to see if that claim holds for any arbitrary ABR algorithms. We performed an analysis in a controlled environment with DASHIF players using different recently designed advanced ABR algorithms. This analysis can decide whether a new ABR algorithm is required to support video streaming for QUIC, or the old one works fine.

The performance of an ABR algorithm can indirectly impact the power consumption of smartphones due to the nature of the radio resource controller (RRC). The RRC depends on three energy state configuration. Incorrect use of radio resources can cause a lot of excess power consumption. The download pattern of the online video streaming system is bursty, i.e., sometimes it downloads data in a burst, and then it pauses for some time. The ABR algorithm decides the pause time and burst size. However, if the ABR algorithms are not aware of the RRC state, the chunk download schedule might unnecessarily consume more energy. The energy consumption by RRC depends on the signal quality as it might increase or reduce the download time of the same size segment. We perform a study using commodity smartphones to understand the relations between energy consumption and different radio signal parameters and video playback. 

%\blue{We choose DASHIF player instead of YouTube itself to control the entire testbed, which is not possible in the case of YouTube. Also, it is not possible to change the ABR algorithm on YouTube as our requirement.}
