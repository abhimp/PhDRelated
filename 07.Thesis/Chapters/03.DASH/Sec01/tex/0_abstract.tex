\begin{abstract}
%\notess{{\bf Outline:} YouTube most important streaming service -- uses DASH recommendations -- existing studies study rate adaptation strategy from the periphery -- we study in-depth and interplay among paramaters -- observe segment length adaptation -- consequentially data wastage is significantly lesser (give value) than reported -- propose predictive model and validate.}
%
YouTube has emerged as the largest player among video streaming services, serving video content for users using DASH.
%
Research studies on various aspects of YouTube, especially its streaming service, abound in the literature.
%
However, these works study YouTube streaming from the periphery, and report results based on their understanding of general DASH recommendations.
%
In this study, we explore in depth YouTube's implementation of the DASH client.
%
We identify important parameters in YouTube's rate adaptation algorithm, and study their roles.
%
In a departure from existing literature, we observe that YouTube opportunistically adapts segment length, in addition to quality level, in response to bandwidth fluctuations.
%
We report that this scheme results in a much lower average data wastage ratio ($0.82 x 10^{-6}$), than reported earlier.
%
We also propose an analytical model, augmented with a machine learning based classifier (with average accuracy of $85.75\%$), to predict data consumption for a playback session in advance.

%\textit{Dynamic Adaptive Streaming over HTTP} (DASH) became very popular in the last decade for adaptive video streaming over time-varying networks, where video quality is tuned based on the available link bandwidth.
%
%Although popular commercial streaming services like YouTube and NetFlix use DASH, there is no way to utilize them as benchmarks for performance analysis of a new DASH implementation, particularly because they are not open-source.
%
%In this paper, we develop a framework for applying reverse engineering methodology over streaming video services, by taking YouTube as a reference.
%
%Through an extensive analysis of YouTube video playback requests collected over $427$ YouTube videos of total time duration $147$ hours $40$ minutes, we figure out the methodology adopted in YouTube for video streaming.
%
%We find that YouTube uses a joint adaptation of video quality and streaming data rate when link bandwidth changes.
%
%We also develop a methodology to compare YouTube with another DASH based service, by exploiting the information transferred via video playback requests.
%
%Our analysis gives a general framework for benchmarking YouTube as a reference DASH implementation that the research community can use for performance analysis of a newly developed DASH service. 
%
\end{abstract}