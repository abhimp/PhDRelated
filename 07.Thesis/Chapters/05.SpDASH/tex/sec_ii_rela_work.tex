\section{Related Work}\label{sec:chap05:related_work}
Existing ABR algorithms use different input parameters with different weightage to decide the optimal bitrates for improving user QoE. Some popular categories of \ac{ABR} algorithms are: 
\begin{enumerate}
    \item Buffer-based - which take the decision based on the playback buffer status~\cite{Spiteri2016,Huang2014}.
    \item Rate based - which estimate the network throughput from  previous chunk downloads to decide the throughput of the current chunk~\cite{Jiang2014,Xu2015}.
    \item Hybrid - which use both the buffer status and rate information to take bitrate decisions \cite{Zou2015}.
    \item QoE metric based - which are designed as  optimization problems for maximizing the user's QoE metric~\cite{Yin2015,Qin2018,Qin2019,Mao2017,Akhtar2018}.
\end{enumerate} 
\cite{Mao2017,Akhtar2018,Sengupta2018,Bampis2018,8816854,Huang2019} are neural network based \ac{ABR} streaming algorithms.  \cite{Mao2017} is one of the earliest papers to use A3C based recurrent neural network for deciding chunk bitrates that optimize QoE. \cite{Akhtar2018} improves \cite{Mao2017} by auto-tuning of parameters. \cite{Sengupta2018} uses reinforcement learning to identify user preferred hotspots in a video and render them at higher bitrates. \cite{Huang2018} uses deep reinforcement learning to achieve a high video quality with reduced transmission bitrates as well as at lower latency. \cite{Huang2019} uses a GAN (Generative Adverserial Network) based deep reinforcement learning algorithm in which agents compete with each other based on a set of rules to arrive at optimal bitrate decisions. \cite{Huo2019} optimizes the QoE of multiple users by learning the diversity in user's QoE requirements using a meta learning approach. \cite{Alt2019} proposes a sparse Bayesian Contextual Bandit Algorithm for designing optimized ABR streaming algorithms in Named Data Networks (NDN). \cite{Bampis2018} uses recurrent neural networks for subjective QoE prediction as a time-series problem.\\
\indent However, these works do not investigate the implementation aspect of the ML-assisted ABR algorithms. In several recent works, edge computing has been used to offload computationally intensive tasks  of IoT or mobile devices. \cite{Guo2019} uses joint computation and communication for proposing an optimal ABR algorithm using mobile edge networking over a fluctuating wireless channel. However, an inherent assumption in the work is the availability of the video at the edge server.\\
\indent In contrast, the \bel\ architecture proposed in our work has been designed to be compatible for any network, such as cellular communication networks or edge networks. In the current cellular network scenario it can be placed at the base-stations or the \ac{CDN} servers. For the future edge networks it can be placed in the edge nodes. Since it is a plug and play module, it can be used over a wide range of networks.

% Buffer-based ABR Streaming: [9] proposes a buffer-occupancy based ABR streaming algorithm which always maintains the playback buffer above 5 seconds duration. Additionally, it chooses the highest bitrate when the buffer has a 15 second cushion. Authors in [13] provide theoretical justification that buffer based algorithms yield better performance than those which need bandwidth occupancy information. In addition they also propose BOLA, a utility maximizing buffer occupancy based ABR video streaming algorithms. The utility function is defined considering all the QoE components.
% Rate-based ABR Streaming:   These algorithms estimate the network bandwidth during the last chunk download and use it as the throughput of the current chunk. The bitrate for the current is chosen from this estimated throughput. An example of a rate-based ABR streaming algorithm is Festive which predicts the throughput of the current chunk as the harmonic mean of the previous five chunks. The highest bitrate below this estimated throughput is chosen as the current bitrate [14]. Authors use a Kalman filter to predict the network throughput in [16].
% Joint Buffer and Rate based ABR video Streaming: [10, 12, 17] propose ABR streaming algorithms which consider both buffer occupancy and current chunk or network throughput to select optimal bitrates for future video chunks. Fast MPC in [10] optimizes the QoE for the next five video chunks based on the past network throughput and buffer occupancy. Robust MPC in [10] further improves fast MPC by exploiting the errors between the actual and the predicted throughputs of the past five chunks. 
% Neural Network based ABR video Streaming:  Of late, neural network based learning algorithms have gained popularity in terms of delivering a high QoE through the choice of optimal bit rates. Authors in [8] propose Pensieve, an ABR streaming algorithm which uses reinforcement learning (RL) algorithm to select optimal bitrates that maximize over a QoE metric.  In [14] authors propose a reinforcement learning based ABR streaming algorithm which renders video segments of higher user interests called hotspots at higher rate than others.
% In [12] authors propose Oboe, which is proposed to work atop algorithms like BOLA, Pensieve, MPC, etc. Oboe first prepares an offline model which sets the parameters of the aforementioned algorithms according to different network states. It then applies change point detection algorithm on network states to tune the parameters of these algorithms
% [8-16] propose ABR streaming algorithms which attempt to improve QoE exploiting various parameters such as playback buffer length, estimated network throughput, etc.

