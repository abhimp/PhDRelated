\section{Motivation}
\subsection{ABR algorithms and interplay with computer networks}
ABR algorithms are an essential part of any adaptive video streaming system, as discussed in the last section. It is assumed that ABR algorithms are fair among themselves and the other type of traffic in most cases. However, during the COVID-19 pandemic, various video streaming services started dropping the highest quality of video from their service to preserve bandwidth available to the ISP as many people started working from home. At that time, the research community started questioning whether it was a fault of the video streaming system? Whether or not the existing ABR algorithms adjust themselves to lower bitrate when the load is high, or there is something wrong with the deployment of those streaming services. While we do not really know the answer, it is clear that the ABR algorithms are very important. The video streaming system needs to adapt to various network conditions. Questions like this motivate us to research the video streaming system.

Google's YouTube is one of the pioneers of video streaming over HTTP(S) and a major online video streaming player. It serves a billion videos to 30 million users daily\footnote{\url{https://blog.youtube/press/}}. To scale the service to support these huge user bases, YouTube has developed several improvements that involve content delivery networks, transport protocol and the adaptation algorithm. No doubt, YouTube gained the attention of the research community to understand the system. In the past, there were several researchers studied YouTube for better understanding. However, most of the work either explore the traffic pattern and QoE properties of YouTube\cite{gill2007youtube,krishnappa2013dashing,wamser2016modeling,wamser2015poster,6757893ieeeexp,7129790ieeeexp} or explore the trade-off between video quality and data wastage\cite{sieber2015cost,seufert2015youtube,sieber2016sacrificing}. Although these studies give an important aspect of YouTube, they failed to explore the internal parameter and streaming strategies. It is crucial to know the adaptation strategies and internal parameters of YouTube to improve the DASH-based system.

Web being Google's primary busyness, it started a project called `Make the Web Faster`\footnote{\url{https://developers.google.com/speed}}. It developed several tools and optimization to load a web page faster. Quick UDP Internet Protocol (QUIC)\cite{langley2017quic} is a product of this project where Google ditched the underlying TCP protocol and provide a new transport protocol based on the UDP. From the release of QUIC, researchers have studied the performance of it in various scenarios, including DASH based video streaming. As the QUIC work differently than TCP, and most of the ABR algorithms used in various video streaming platform are designed for TCP as the transport protocol, those ABR algorithms might not work well with QUIC. Research works like \cite{bhat2018improving,van2018empirical}, studied QUIC performance on ABR algorithm, found that buffer-based ABR algorithms are not well suited for QUIC. However, those works are performed using very old ABR algorithms, and studies are mostly simulation-based. Also, these works do not provide any insight into the root cause of the differences.

\subsection{Energy efficiency in online video streaming}
As per YouTube's report, users love to play online videos on a smartphone. It is also true for other streaming platforms. While most of the smartphone one-a-days have a huge battery capacity, it is still not enough to play videos for a long duration as video streaming is inherently energy-consuming. The uneven distribution of LTE networks, clustered crowds, and non-optimized video app can make these things even worse. Research in \cite{10.1145/2910018.2910656} shows that there is plenty of places to save energy by optimally configuring the streaming parameters like segment length, maximum buffer length, and segment download scheduling. These researches indicate that the interplay between the cellular network and the device's radio controller plays are a significant role in the energy efficiency of online adaptive video. It is important to learn various cellular network parameters and make informed adaptive decisions based on the learned parameters to make the adaptation more energy efficient.

\subsection{Live Streaming}
Live streaming is becoming an alternative to TV as the live broadcasts of most mega-events are broadcasted over the Internet using online live streaming. As more users are shifting towards online live streaming, it is becoming difficult to scale the mega-events' live streaming system. The existing literature proposes solutions by using Peer-to-Peer (P2P) networks to scale the live video streaming as players are in sync. However, P2P based live streaming suffers as most of the ISP allocate very less uplink bandwidth than downlink bandwidth to their subscriber. 
When a P2P network span over the Internet, traffic needs to go through the Internet, and this traffic will count towards the user's data connection. So, users may not take it lightly. However, in the case of mega-events, many users stay in clusters and share a common Internet backbone. One service can try to exploit the internal network to extend the service and provide better quality with lower overhead to the server and the ISP.