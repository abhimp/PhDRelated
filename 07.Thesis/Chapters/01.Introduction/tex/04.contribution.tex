\section{Contribution}
\subsection{Adaptive Bitrate Streaming over Today's Internet: A Case Study of YouTube}
\begin{enumerate}
	\item We illustrate a methodology to study YouTube's adaptive streaming behavior in-depth (\S\ref{chap03s1:sec:experiments}) -- we identify and closely study the interplay among important parameters enabling this streaming algorithm (\S\ref{chap03s1:sec:parameters}).
	\item Our experiments reveal that YouTube adapts the {\it segment length} parameter before attempting to adapt video resolution -- a phenomenon not reported in the literature (\S\ref{chap03s1:subsec:seglength}).
	\item We observe that segment length adaptation leads to much lower values of data wastage on average, than reported by prior studies.
	\item We propose an analytical model, augmented with a machine learning based classifier, which enables prediction of data consumption for an initial playback video quality when it is possible to estimate the network conditions a priori using existing mechanism like~\cite{Zou2015}  (\S\ref{chap03s1:sec:model}).
\end{enumerate}

\subsection{Transport Protocols and Mobility Choices for ABR Streaming: Performance vs Energy Efficiency}
\begin{enumerate}
	\item Similarly to the previous chapter, we first conduct a study over YouTube to analyze YouTube ABR performance over TCP and QUIC. 
	\item We then perform a study over a controlled environment, to explore how different modern ABR algorithms like BOLA, MPC, Pensieve, etc. work over QUIC in comparison to TCP. 
	\item Finally, to understand the impact of ABR streaming over the energy consumption of a mobile device, we conduct a through experiment using commodity smartphones and commercial cellular network connections. 
\end{enumerate}

\subsection{EnDASH - A Mobility Adapted Energy
Efficient ABR Video Streaming for
Cellular Networks}
\begin{enumerate}
	\item With the observation from previous chapters, we developed a throughput prediction engine to predict average throughput for a short future.
	\item We design a ABR algorithm based on the throughput prediction algorithms and instantaneous parameters to reduce the power consumption by optimally adjust the playback buffer lengths.
\end{enumerate}

\subsection{Federated Adaptive Bitrate Live
Streaming over Locality Sensitive
Playback Coalitions}
\begin{enumerate}
	\item We identified possibility of exploiting local network to reduce server load during live streaming.
	\item The work proposes a novel way to share segments among users fairly and efficiently by form coaliation among nearby and similar type of players.
	\item It also proposes a novel algorithm to select bitrate to increase QoE of entire coalition simulatenously instead of itself only.
	\item The proposed system, improved QoE for all the player while reducing the network usages with compared to existing system.
\end{enumerate}