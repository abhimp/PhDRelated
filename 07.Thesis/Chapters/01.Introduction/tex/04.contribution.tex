\section{Contributions}
This thesis consists of four contributory chapters. The significant contributions of the individual chapters are given below. 

\subsection{Adaptive Bitrate Streaming over Today's Internet: A Case Study of YouTube}
In this chapter, we perform reverse engineering over the YouTube \ac{ABR} streaming algorithm to find out how it works and what are the parameters that impact the performance of \ac{ABR} streaming over a large-scale Internet scenario. The contributions in this chapter are as follows. 
\begin{itemize}
	\item We illustrate a methodology to study YouTube's adaptive streaming behavior in-depth -- we identify and closely study the interplay among important parameters enabling this streaming algorithm.
	\item Our experiments reveal that YouTube adapts the {\it segment length} parameter before attempting to adjust video resolution -- a phenomenon not reported in the literature.
	\item We observe that segment length adaptation leads to much lower data wastage values on average than the ones reported by prior studies.
	\item We propose an analytical model, augmented with a machine learning-based classifier, which enables prediction of data consumption for an initial playback video quality when it is possible to estimate the network conditions a priori using existing mechanism like~\cite{Zou2015}.
\end{itemize}

\subsection{Transport Protocols and Mobility Choices for ABR Streaming: Performance vs. Energy Efficiency}
In the second contributory chapter of the thesis, we analyze various state-of-the-art \ac{ABR} techniques under different transport protocols and mobility choices. This analysis gives us an idea about how \ac{ABR} algorithms work in diverse scenarios, evident over today's Internet. The contributions in this chapter are summarized below. 
\begin{itemize}
	\item Like the previous chapter, we first conduct a study over YouTube to analyze YouTube \ac{ABR} performance over two different transport protocols -- \ac{TCP} and \ac{QUIC}.
	\item We then perform a study over a controlled environment, to explore how different modern \ac{ABR} algorithms like BOLA~\cite{Spiteri2016}, MPC~\cite{yin2015control}, Pensieve~\cite{mao2017neural}, etc. work over \ac{QUIC} in comparison to \ac{TCP}.
	\item Finally, to understand the impact of \ac{ABR} streaming over the energy consumption of a mobile device, we conduct a thorough experiment using commodity smartphones and commercial cellular network connections under different mobility scenarios. We observed that arbitrary data download through \ac{ABR} may impact the smartphone's energy consumption behavior when some \ac{OTT} media application runs for streaming a video. 
\end{itemize}

\subsection{EnDASH - A Mobility Adapted Energy Efficient ABR Video Streaming for Cellular Networks}
This chapter develops an energy-efficient video streaming algorithm while preserving the \ac{QoE} performance of the \ac{ABR} mechanism. The salient contributions of this chapter are as follows. 
\begin{itemize}
	\item With the observation from previous chapters, we developed a throughput prediction engine to predict average throughput for a short future.
	\item We design an \ac{ABR} algorithm based on the throughput prediction algorithms and instantaneous parameters to reduce the power consumption by optimally adjust the playback buffer lengths.
	\item We performed thorough experiments in an emulated environment to show the performance of the proposed mechanism, called \textit{EnDASH}. We observe that the proposed mechanism significantly saves energy consumption during video streaming with marginal compromise in the overall \ac{QoE}. 
\end{itemize}

\subsection{Federated Adaptive Bitrate Live Streaming over Locality Sensitive Playback Coalitions}
In the final contributory chapter of the thesis, we focus on \ac{ABR} for the live streaming of videos and develop an architecture to support efficient adaptive live streaming over the Internet. The contributions of this chapter are as follows. 
\begin{itemize}
	\item In this chapter, we have identified the possibility of exploiting the local network to reduce server load during live streaming, which improves the \ac{QoE} for the streaming application.
	\item The work proposes a novel way to share segments among users reasonably and efficiently by forming a coalition among nearby and similar types of video players.
	\item It also proposes a novel algorithm to select bitrates adaptively to increase the \ac{QoE} of the entire coalition simultaneously, instead of individual \ac{QoE} only.
	\item We have evaluated the proposed system over an emulation environment, and we observe that our system can significantly improve the live streaming \ac{QoE} compared to other baselines. 
\end{itemize}