The Internet, which connects systems all over the globe, has become part of the modern lifestyle. The basic service like telephony to complicated procedure telesurgery nowadays require the Internet. While it provides various services, video streaming is the most prominent one. According to Cisco VNI predictions\footnote{\url{https://www.cisco.com/c/dam/m/en_us/network-intelligence/service-provider/digital-transformation/knowledge-network-webinars/pdfs/1213-business-services-ckn.pdf}}, by 2022, almost half of the Internet connected device going be video capable, which will contribute 84\% of all Internet traffic. Also, the high definition of video demand will increase by many folds as more connected devices will be 4K capable. To satisfy this massive growth in video streaming demand, every part of the network stack needs to be optimized, and the online video streaming system is no exception. There is 3 type of online video streaming system. These are a) interactive video streaming, b) live video streaming, and c) static or video-on-demand (VoD). Among these three categories, live, and VoD streaming contributes to significant traffic on the Internet. While services like YouTube, twitch support both these types of video streaming, NetFlix, PrimeVideos are primarily concentrated in the VoD service. 

In earlier days, online video streaming used to use more dedicated protocols like RTP, SRTP. These protocols are push-based, which means servers push the video content towards the clients, and a client almost has no control over it. Due to being pushed based protocol, most of the proxies and NAT boxes usually block them (intentionally and unintentionally). To overcome these issues, YouTube started delivering video over HTTP protocol in 2005. As the HTTP is used to serve the world-wide-web (WWW), it was widely accepted by NAT boxes and proxies, and video data can go pass through without any problem. Soon, video stream over HTTP become a defacto standard for the web-based video player. Initially, the HTML did not support any video playback mechanism, so Adobe flash player was used as it supported the progressive video file download and playback of partially downloaded video files\cite{gill2007youtube}. HTML embedded audio-video playback capabilities were introduced in HTML5, and all the web-based online video streaming services moved towards the HTML5 video playback.

In 2009, Apple released HTTP Live Streaming service for their QuickTime video player. It includes a feature, adaptive bitrate, where video quality can be adapted based on the observed network quality to avoid rebuffering. Soon almost all the services started service adaptive video streaming. The different organizations have developed different techniques for adaptive video streaming over HTTP. Among them, MPEG-DASH\cite{ISO/IEC23009-1:2019} is notable as it has an open-source implementation by DASH-IF. HTML5 has added an experiment extension called Media Source Extension (MSE)\footnote{\url{https://www.w3.org/TR/media-source/}}, supporting the adaptive video. 

While video streaming is very popular over the web, it is also popular with smartphone users. Most smartphone operating systems, including Android, IOS, and Windows, support adaptive video streaming in both HTML5 based video players through the browser and native video player for the application. Furthermore, most of the online video services started providing an application to run on such smartphones as well as the support for their website in the mobile version.

Currently, most the the streaming providers support adaptive video streaming. Any adaptive streaming system runs an adaptive bitrate (ABR) algorithm to adapt video quality based on the network quality. ABR algorithms are the most critical part of a video streaming system as the quality of experience (QoE) depends on them. QoE is a combination of three parameters\cite{yin2015control} i) overall quality, ii) quality fluctuations or lack of smoothness, and iii) rebuffering. Any ABR is supposed to improve QoE by maximizing overall quality and minimizing quality fluctuations and rebuffering. At the inception of adaptive video streaming, the ABR algorithm was simple; it matched the video bitrate to the observed throughput\cite{5677508,10.1145/1943552.1943575,10.1145/1943552.1943574}. However, this technique causes frequent fluctuation in bitrate, which can cause irritation to the users. So, the researcher started improving the ABR algorithm with various techniques. Over the decade, throughput based algorithm improved by buffer-based\cite{Spiteri2016,10.1145/2910017.2910596,7393865} and later on the hybrid of both throughput and buffer based\cite{7247436,140405,yin2015control,10.1145/2670518.2673877}. Finally, to improve it even more, machine learning-based\cite{mao2017neural,Akhtar2018,9155492} algorithms are developed.

The online video streaming system is very popular, and adaptive video streaming over HTTP is its core. The adaptive video streaming is network friendly as it can reduce its bandwidth requirement by choosing a lower bitrate allowing other applications to go through a shared bottleneck. When multiple players share a common bottleneck link, they automatically adjust their video quality to allow other players to stream with reasonable quality. Over time, all the player gets an almost fair share of the bottleneck link.