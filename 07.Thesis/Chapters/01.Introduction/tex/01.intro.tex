The Internet, which connects systems all over the globe, has become part of the modern lifestyle. The basic services like telephony to complicated procedures like telesurgery nowadays require the Internet. While it provides various services, video streaming is the most prominent one over today's Internet. According to Cisco Visual Networking Index (VNI) predictions\footnote{\url{https://www.cisco.com/c/dam/m/en_us/network-intelligence/service-provider/digital-transformation/knowledge-network-webinars/pdfs/1213-business-services-ckn.pdf} (Access: \today)}, by 2022, almost half of the Internet-connected devices are going to be video capable, which will contribute 84\% of all the Internet traffic. Also, the high definition video demand will increase by many folds, as more connected devices will be 4K capable. To satisfy this massive growth in video streaming demand, every parts of the network, starting from the hardware to the software, need to be specially optimized for supporting massive-scale online video streaming. There are three different types of online video streaming systems which are viable over the Internet -- (a) \textit{interactive video streaming}, (b) \textit{live video streaming}, and (c) \textit{static} or \textit{video-on-demand} (VoD). Among these three categories, live, and VoD streaming contributes to significant traffic on the Internet, as they mostly demand high-definition videos. While services like YouTube, Twitch, etc. support both these types of video streaming, NetFlix, Amazon Prime Videos, etc. are primarily concentrated on supporting the VoD service. 

In earlier days, online video streaming used to utilize more dedicated protocols like \textit{Real Time Protocol} (RTP), \textit{Secure Real Time Protocol} (SRTP), \textit{Real Time Streaming Protocol} (RTSP), etc. These protocols are push-based, which means the servers push the video content towards the clients, and a client almost has no control over it. Due to being pushed-based protocol, most of the network middleboxes like proxies and network address translation (NAT) usually block them (intentionally and unintentionally). To overcome these issues, YouTube started delivering video over Hypertext Transfer Protocol (HTTP) in 2005. As the HTTP is used to serve the World Wide Web (WWW), it was widely accepted by NAT boxes and proxies, and video data can pass through without any problem. Soon, video streaming over HTTP became a defacto standard for the web-based video player. Initially, the HTML did not support any video playback mechanism, so Adobe flash player was used as it supported the progressive video file download and playback of partially downloaded video files~\cite{gill2007youtube}. HTML embedded audio-video playback capabilities were introduced in HTML5, and all the web-based online video streaming services moved towards the HTML5 video playback.

In 2009, Apple released HTTP Live Streaming service for their QuickTime video player. It included a feature, adaptive bitrate (ABR), where video quality can be adapted based on the observed network quality to avoid rebuffering. Soon almost all the services started service adaptive video streaming. The different organizations have developed different techniques for adaptive video streaming over HTTP. Among them, \textit{Dynamic Adaptive Streaming over HTTP} (DASH) by MPEG~\cite{ISO/IEC23009-1:2019} is notable as it has an open-source implementation by DASH Industry Forum (DASH-IF). HTML5 has added an experimental extension called Media Source Extension (MSE)\footnote{\url{https://www.w3.org/TR/media-source/} (Access: \today)}, supporting the adaptive video. 

%While video streaming is very popular over the web, it is also popular with smartphone users. Most smartphone operating systems, including Android, IOS, and Windows, support adaptive video streaming in both HTML5 based video players through the browser and native video player for the application. Furthermore, most of the online video services started providing an application to run on such smartphones as well as the support for their website in the mobile version.

Currently, most the the streaming providers support adaptive video streaming. Any adaptive streaming system runs an adaptive bitrate (ABR) algorithm to adapt video quality based on the network quality. ABR algorithms are the most critical part of a video streaming system as the \textit{quality of experience} (QoE) depends on them. Video QoE is a combination of three parameters~\cite{yin2015control}, -- (i) the overall quality, (ii) quality fluctuations or lack of smoothness, and (iii) rebuffering. Any ABR is supposed to improve QoE by maximizing the overall quality and minimizing the quality fluctuations and the rebuffering. At the inception of ABR video streaming, the ABR algorithm was simple; it used to match the video bitrate to the observed throughput~\cite{5677508,10.1145/1943552.1943575,10.1145/1943552.1943574}. However, this technique causes frequent fluctuations in the bitrate, which can cause irritation to the users. So, the researchers started improving the ABR algorithms with various techniques. Over the decade, the throughput-based algorithms were improved by the buffer-based~\cite{Spiteri2016,10.1145/2910017.2910596,7393865} and later hybrid -- a combination of throughput and buffer-based algorithms~\cite{7247436,140405,yin2015control,10.1145/2670518.2673877}. Very recently, to improve the QoE even more, machine learning (ML)-based\cite{mao2017neural,Akhtar2018,9155492} ABR algorithms have been proposed. However, as the number of streaming media users grows day by day, and with having numerous video streaming apps available over smartphones, new challenges are being observed, demanding several optimizations in the current ABR streaming systems. 

%The online video streaming system is very popular, and adaptive video streaming over HTTP is its core. The adaptive video streaming is network friendly as it can reduce its bandwidth requirement by choosing a lower bitrate allowing other applications to go through a shared bottleneck. When multiple players share a common bottleneck link, they automatically adjust their video quality to allow other players to stream with reasonable quality. Over time, all the player gets an almost fair share of the bottleneck link.