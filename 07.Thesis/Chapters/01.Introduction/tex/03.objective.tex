\section{Objectives of the Thesis}
The objective of this thesis are as follows. 

\subsection{Understanding Adaptive Streaming Services over Today's Internet}
As we mentioned, almost all the over-the-top (OTT) streaming media services over the Internet has adopted DASH and use some form of ABR algorithms at the client application. Given the context and importance of video streaming services, it is necessary to understand how they work and how the interact with the web and other traffics over the Internet. Also it is necessary to understand the interactions between the underlying networking protocols, like QUIC, and the ABR streaming services running as Internet applications. With this requirements in mind, the first objective of this thesis is to analyze YouTube streaming media services to explore the ABR algorithms and how it interacts with the various networking protocols. Although the adaptation mechanism in YouTube is based on DASH, we want to find the differences between the conventional DASH principles and the streaming algorithm used by YouTube, As YouTube is a closed source commercial OTT application and uses proprietary services, the only way to understand the ABR mechanism of YoUTube is by reverse engineering the system based on observations of its behavior. 

\subsection{Explore the Impact of Network Protocols and Client Mobility on the Performance of ABR Techniques}
The performance of the ABR algorithms depends on the observed network quality; however, it highly depends on the underlying transport layer, TCP, until Google developed QUIC. Our next objective is to explore the impact of the underlying transport protocols, such as TCP and QUIC, on the behavior of various modern ABR techniques~\cite{Spiteri2016,Mao2017}. We further envision to explore the energy-efficiency of ABR streaming algorithms, as majority of the popular OTT media services run as a client application over smartphones, and user prefer to watch videos while in transit. Consequently, the impact of mobility on smartphone energy-efficiency due to the network traffic generation patterns by the ABR algorithms is an important field of study. Our objective is to explore the same over diverse networking environments.  


\subsection{Energy Efficient Video Streaming over Smartphones}
Power management in a smartphone is very crucial that includes video streaming, especially when users are moving. So, we aim to develop an ABR algorithm, especially for smartphone base video streaming, to reduce battery consumption while maintaining reasonable QoE. We utilize the radio-related information available in a smartphone. Also, other information might be useful, such as GPS location, device speed, battery information. We want to find a correlation between these parameters in network condition and schedule the segment downloading in such a way so that it download most of the segments with the least energy consumption. The primary objectives can be summarized as follows:\\
a) Find the correlation between radio, battery, and other physical parameters of a device with the network quality.\\
b) Utilize these parameters to develop an ABR algorithm to reduce energy consumption while video streaming.\\
c) The ABR should provide reasonable QoE, although it might sacrifice the finest QoE to preserve energy.

\subsection{Federated Adaptive Bitrate Live
	Streaming over Locality Sensitive
	Playback Coalitions}
This work aims to reduce Internet and server load while maintaining high QoE during live streaming by exploiting the feature of synchronous playback and locality information. We want a player to find and share segments with other players in the locality during live streaming. We do not want any players to upload segments through the Internet as most users have very poor uplink and uploaded data count towards their data usage. Also, upload through the Internet means other users have to download segments from the Internet, which is equivalent if not worse than download from the CDN as per the players' perspective. It is also important that all the players contribute toward the segment downloading from CDN. It ensures fairness among players, even if they have different data plans. The primary objectives are as follows: \\
a) Exploit synchronous playback mechanism of live streaming and share segments among the player.\\
b) No data should be uploaded to the Internet during sharing.\\
c) Every player which are getting segments from other players should also share segment by downloading it directly from the CDN. \\
d) Every player should work towards improving QoE for every other player.