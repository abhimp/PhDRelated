\section{Organization of the Thesis}
Henceforth, this synopsis report of the thesis is organized as follows. Section 2 gives a broad overview of the existing literature and its limitations. Section 3 highlights our contributions in exploring the YouTube ABR mechanism over today's Internet and the importance of various parameters that govern the YouTube streaming algorithm. In Section 4, we summarize our observations in exploring the protocol and mobility choices for designing ABR algorithms while keeping energy-efficiency as an objective. Section  5 discusses EnDASH, and adaptive streaming mechanism, which considers energy-efficiency as a goal while developing the ABR. Section 6 constitutes the summary of FLiDASH -- an adaptive live streaming mechanism to improve the QoE while reducing the network overhead. Finally, Section 7 gives the thesis's organization, and Section 8 concludes the report by highlighting the publications out of the thesis in Section 9.

%In this section, we provide a brief description of the organization of the thesis.
%\begin{itemize}
%	\item {\bf Chapter 2} provide details of online video streaming system and different ABR algorithm developed in the past. Moreover, it iterates over the merits and demerits of the proposed solution to the problem broadly linked with this thesis.
%	\item {\bf Chapter 3} present the study on YouTube video streaming service and analyses the adaptation technique taken by YouTube.
%	\item {\bf Chapter 3} performs a study to analyze the performance difference between two transport protocols, TCP and QUIC, on the video streaming system, both YouTube and DASH. We further analyze the energy consumption by a smartphone during online video streaming in various scenarios.
%	\item {\bf Chapter 4} present a novel ABR algorithm EnDASH to reduce energy consumption during the online streaming by optimizing the segmented schedule in buffer length.
%	\item {\bf Chapter 5} proposes a DASH based video streaming system which exploits the locality information to improve the video QoE and decrease the network usage by sharing segment among local players. It also describes a novel approach to form a coalition among local players and decide bitrate for the entire coalition.
%	\item {\bf Chapter 7} concludes the thesis by summarizing our contributions and listing down the possible future directions.
%\end{itemize}
