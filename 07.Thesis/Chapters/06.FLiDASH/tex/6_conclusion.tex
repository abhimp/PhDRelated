\section{Discussion and Conclusion}
In this work, we develop a middlebox-free collaborative adaptive live streaming system which improves the overall QoE while reducing the network traffic. The architecture utilizes a federated platform where the streaming players form coalitions and schedule video downloads in a distributed way without the help of any Internet middleboxes. The proposed architecture has been implemented and evaluated over a thorough emulation platform, and we observe significant improves in the overall QoE with a reduction in the load at the live streaming server. 

A real-world implementation of {\our} requires the support of a proximity server and installation of the playback client in the end-user's device. It can be noted that the currently available commercial client applications (like YouTube, Twitch clients) need to have {\our} support, although no changes are required at the content-server or ISP side. Further, the system does not use any Internet-middlebox for coalition formation. We can also address the issues faced by existing internet middleboxes, like when the clients are behind a NAT, by using a \textit{Session Traversal Utilities for NAT} (STUN)~\cite{rfc5389_stun} or a \textit{Traversal Using Relays around NAT} (TURN)~\cite{rfc5766_turn} server. In this aspect, {\our} provides a more deployment-ready solution compared to ICN-DASH~\cite{ICN-DASH} or Multicast ABR~\cite{multicastAbrCablelabs}.

We believe that {\our} has the potential for deploying a highly-scalable architecture for mass-scale live streaming of videos while incurring low overhead to the backbone network. Indeed, such a system can be very useful for various developing countries where live streaming of video contents has significantly increased in the recent years, although the network backbone infrastructure is yet to be matured~\cite{kiedanski2019youtube}.
