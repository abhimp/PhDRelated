\section{Related Work}
There are few works in the literature like \textit{DASH over Information Centric Networking} (ICN)~\cite{ICN-DASH} and \textit{Multicast ABR}~\cite{multicastAbrCablelabs}, which use adaptive bitrate in a collaborative setup. All the segments of a DASH video contain a unique name through the URL. ICN provides a way to get the contents based on a name, and thus the DASH video segments can be routed according to the name. In case of a ICN, if the video players are behind a common ICN gateway, the gateway can cache the DASH video segments based on the unique name, and thus, can reduce duplicate delivery of contents. However, ICN needs significant changes in the network architecture including its routing policies, and thus, is not readily deployable on the existing networks. In 2016, Cablelabs introduced multicast ABR (M-ABR)~\cite{multicastAbrCablelabs} to provide the same video content to a large group of viewers while ensuring low network overhead via IP multicast. In M-ABR, all the ISPs maintain an M-ABR device that gets connected to the original video server. The M-ABR device receives the data from the video content server and then uses IP multicast to forward the content to the end-users. However, in this architecture, the content provider needs to push a middlebox (M-ABR server) to the ISP. Further, there needs to be a software change at the client devices. There exist few works in the literature to take care of the ABR selection over a collaborative setup, however, in these systems, the video players need to coordinate with an Internet middlebox that works as a central controller, such as a tracker~\cite{detti2016tracker}, a software controller~\cite{khalid2019sdn} or the cloud~\cite{payberah2012clive,wang2014migration}. This limits the scalability as all the video players need to coordinate with the controller for each ABR decisions which are taken for every video segment requests. In contrast to such existing approaches, {\our} works without the support from any such Internet middleboxes. 
