\section{Experimental Setup}
\label{chap03sec:experiments}

In this study, experiments are conducted with two broad objectives: (1) Observing YouTube's adaptive streaming behavior under different network conditions, and (2) Extracting protocol level parameters from granular video playback data under controlled network conditions, and deducing their individual roles.

{\bf Experimental setup for observing streaming behavior:} In the preliminary experiment, we play a YouTube video with frequent screen changes (video title: {\em ``The Division Walkthrough Gameplay Part 1 -- The Virus (PS4 Xbox One)''}, video ID: {\em b80ShWk\_Aro}, URL: \url{https://www.youtube.com/watch?v=b80ShWk_Aro}, video duration: $34$ min. $22$ sec.) by varying the link bandwidth using a dynamic throttling mechanism.
We refer to this video as our {\it sample video} throughout the rest of the paper.
Link throttling is enabled using the Unix library \texttt{NetFilterQueue}\footnote{\url{http://www.netfilter.org/projects/libnetfilter_queue/}}, which is a user-space library with an API to handle packets queued by the kernel packet filter.
Based on this library, we develop a traffic shaper to control link bandwidth.
We continuously monitor the bandwidth experienced at the browser, and ensure that the data is collected at intended bandwidth values.
During the video playback, we capture video data packets using the Unix tool \texttt{tcpdump}; from these packet traces, we extract the amount of video data transferred from the YouTube server to the client, with respect to time.
Additionally, we note the resolution in which the video is rendered, w.r.t. time.

\begin{table}[!t]
 \small
 \centering
 \caption{\small{Information regarding YouTube videos used in the experiments}}
 \label{table:chap03:statvid}
 \begin{tabular}{|c|c|c|}
  \hline 
  \textbf{Video Size} & \textbf{Number of Videos} & \textbf{Total Playback Duration} \\
  \hline \hline 
  $<10$ mins & $195$ & $19$h $44$m \\
  \hline 
  $10-30$ mins & $104$ & $26$h $07$m \\
  \hline 
  $30-60$ mins & $122$ & $94$h $33$m \\
  \hline 
  $>60$ mins & $30$ & $37$h $15$m \\
  \hline 
 \end{tabular}
\end{table}
{\bf Experimental setup to identify parameters and their interplay:} Since DASH works on top of HTTP, the video playback information and the requested video quality are embedded in HTTP request/response headers.
As part of the {\em developer tools} for Mozilla Firefox, a {\em network monitor} is available, where the browser dumps information regarding all requests made by the current page -- HTTP-request, HTTP-response, request/response time, link speed, etc.
The entire information can be exported to {\bf H}TTP {\bf AR}chive (HAR) files, which are in the JSON\footnote{\url{http://www.json.org/}} file format.
We develop an automated tool called {\em AutoHARExporter} using \texttt{Selenium}\footnote{\url{http://www.seleniumhq.org/}} to capture a HAR from the browser with the help of \texttt{har\_export\_trigger} (version 0.5.0-beta) Firefox plugin.
This tool automatically opens a Firefox browser, loads a YouTube video (in the format {\tt \url{https://youtube.com/watch?v=<video id>}}), waits for the video to finish, and finally saves the HAR and other information to the disk.
During video playback, we also control network bandwidth (using the throttling mechanism described earlier) by progressively increasing and then dropping it from a given set of bandwidth levels ranging from $200$ Kbps to $2400$ Kbps, in a step of $200$ Kbps.
Each level of bandwidth is kept fixed for $200$ seconds. 
We repeat the experiment for $\sim500$ videos (detailed statistics shown in Table~\ref{table:chap03:statvid}), the list of which is collected in advance by crawling the YouTube website.

