\Abstract{
\addcontentsline{toc}{chapter}{Abstract}
\fontsize{12}{13}\selectfont %\dropping[0cm]{2}

In the era of low-cost, high-end smartphones, smart TVs, and cheap 4G-LTE data plans, video streaming becomes an essential service on the Internet. Most online video streaming services are adaptive video streaming, and adaptive bitrate algorithms play an important role in maintaining Quality of Experience, data usages, and energy consumption. From the inception in 2009, several researchers proposed different ABR algorithms and changes in the architecture to improve the QoE, energy efficiency. However, there is much to improve in the video streaming system to support more advanced devices, scale to millions, and reduce energy requirements. In this work, we have tried to analyze the existing services and suggested a few improvements.

Video streaming services contribute most of the Internet traffic globally and make every player on the Internet to adapt and improve their technologies to support the growing users. It is so massive that during the COVID'19 pandemic, streaming providers decided to drop their top-quality to relieve Internet service to get paced to support the sudden growth in work-from-home users. It also raises a question to the research community why ABR algorithms are not enough to handle excess bandwidth requirements. Though we do not know the question, it provides perspective regarding this domain's importance and motivates many to dig deep. This type of issue motivates us to understand how the existing system works what can be done to improve it.

YouTube is a major video streaming provider and the pioneer of web-based video streaming service. As per their press release, it serves a billion videos to 30 million users every day. It is interesting the see how YouTube serve adaptive video to users. As the literature lacks the YouTube adaptive streaming system's details, we perform experiments to reveal it. We used our homegrown tools to get request-response pair during video playback in a browser for ~500 YouTube videos and analyzed it. Our analysis finds that YouTube uses an opportunistic upscale and conservative downscale of video quality based on network quality and adapts segment length before changing quality. We also observed that YouTube sometimes tries to download the same segment of different quality, which can cause data wastage. However, our calculation reveals that data wastage is negligible.

By 2017, Google made Quick Internet UDP Connect or QUIC as the default transport protocol for all its services from the Chrome browser. We experiment to find out its impact on the video streaming services, especially on the ABR algorithms. The experiment has two parts. First, we analyzed YouTube and found that the QUIC provides better video quality with lower rebuffering than TCP. In the second part, we compare popular ABR algorithms' performance over QUIC and TCP and found that these ABR algorithms are not well suited for QUIC. When we dig deep, we realized that problem lies in the userspace implementation of QUIC. In this work, we also wanted to know the impact of video streaming during mobility on energy efficiency. We studied power consumption in various commodity smartphones while playing YouTube videos with different mobility states in different cities. In this study, we observed that YouTube does not care about the instantaneous throughput or the radio state, which can cause a severe impact on power consumption.

With last observations, we design the EnDASH ABR algorithm that reduces power consumption during the video streaming by carefully scheduling the segment downloading based on the various device-related parameters, including cellular radio, battery, and device movement. EnDASH divides the entire playback session into slots. At the beginning of each slot, it first estimates the estimate the average throughput and then selects the maximum playback buffer length based on the slot based on the observations last slot. The buffer change in playback buffer length triggers the rescheduling in segment downloading. The EnDASH also has a module that decides the video quality for each segment, which further reduces power consumption. It requires three different learning engines, a) Throughput estimation engine, b) Buffer length prediction engine, and c) quality selection engine. The throughput estimation engine is based on random forest reinforcement learning. The buffer length prediction engine and quality selection engine are designed based on an actor-critic based recurrence neural network. We find that EnDASH provides better energy efficiency during evaluation than the existing ABR algorithm with a slight compromise in the QoE.

After saving energy, we decided to save some data usages. So, we developed FLiDASH, a federated adaptive live streaming system that exploits players' locality by forming coalitions. We observed that mega-event live streamers are localized, and most of the cases are connected via a local network. However, the current setup does not exploit the local network. Instead, every player downloads the segment on their own. FLiDASH utilizes the local network to find out players that can be communicated without involving the Internet. Then it forms a coalition among players that have similar quality targets. Every player in a coalition contributes to QoE's improvement by downloading based quality video it can download. We use the moving leader concept, where the leader is responsible for downloading a segment and deciding the leader for the next segment. These two algorithms work in a way so that a player with the lowest download speed gets to download the furthest segment in the long run. Our evaluation reveals that FLiDASH can improve the QoE for a coalition while reducing the server load and the network load. We also find that every player gains QoE and average quality by using FLiDASH than using any other ABR algorithms.

In the thesis, we analyze YouTube's adaptive streaming behavior and find interesting insights. We also analyzed the energy efficiency of YouTube and ABR's compatibility with different ABR algorithms. With study, we develop EnDASH to save energy in smartphones during video playback. Later we save data usages and server load by sharing segments among video players. In the future, we have to find an ABR algorithm for better compatibility with QUIC and TCP. It is also important to find a way to deploy a machine learning-based ABR algorithm, which is difficult now due to resource and technology requirements.

\textbf{Keywords:}
}

%%%%%%%%%%%%%%%%%%%%%%%%%%%%%%%%%%%%%%%%%%%%%%%%%
