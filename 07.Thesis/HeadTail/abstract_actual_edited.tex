\Abstract{
\addcontentsline{toc}{chapter}{Abstract}
\fontsize{12}{13}\selectfont %\dropping[0cm]{2}

In the age of low-cost, high-end smartphones, smart TVs, cheap data plans and Covid-19 pandemic, video streaming over the Internet has emerged as an essential service, for online meetings as well as for entertainment. Most online video streaming services are adaptive in nature wherein adaptive bitrate algorithms are used for maintaining high Quality of Experience, improving data usage as well as energy consumption. From the conception of ABR streaming algorithms in 2009, several researchers have proposed different versions of these algorithms and changes in the architecture to facilitate their implementation. However, there is much room left for improvement of the video streaming system so as to improve their scalability to support millions of devices while improving their energy management This thesis focusses on investigating the properties of different existing ABR video streaming services and also suggests a few improvements.\\
\indent Video constitutes the largest fraction of the global mobile traffic volume. To support the continuous increase in the number of video subscribers every player on the Internet is required to adopt improved technologies. For example, during the COVID'19 pandemic, OTT streaming providers had decided to drop their top-qualities to reduce the amount of Internet traffic so as to support the higher number of work-from-home users. This raises the doubt whether ABR algorithms alone are not enough to handle the excess bandwidth requirements. Although the answer remains elusive, the doubt establishes the importance of the domain and motivates one to dig deeper to understand how the existing system works and what are the typical measures that can be adopted for its improvement.\\
\indent YouTube is one of the major video streaming providers and the pioneer of web-based video streaming services. As per their recent press release, YouTube serves nearly a billion videos to 30 million users every day. Hence, a deep dive into how YouTube serves video to users adaptively is expected to yield answers to several intuitions. As the literature lacks an in-depth analysis of YouTube's adaptive streaming service, in this work we have first performed experiments to understand the same. We have used our homegrown tools to get a request-response pair during video playback in a browser for ~500 YouTube videos and have subsequently analyzed them. Our analysis shows that YouTube uses an opportunistic upscale and conservative downscale of video quality based on network quality and adapts the segment length before changing quality. We also observed that YouTube sometimes tries to download the same segment at different qualities, which can cause data wastage. However, our calculation reveals that the ensuing data wastage is negligible.\\
\indent Currently Google uses Quick Internet UDP Connect or QUIC as the default transport protocol for all its services from the Chrome browser. In the second  work, we experiment to find out its impact on the video streaming services, especially on the ABR algorithms. The experiment has two parts. First, we have observed YouTube and found that the QUIC provides better video quality with lower rebuffering than TCP. In the second part, we have compared popular ABR algorithms' performance over QUIC and TCP and found that these ABR algorithms are not well suited for QUIC. When we dig deep, we realized that problem lies in the userspace implementation of QUIC. In this work, we also wanted to know the impact of video streaming during mobility on energy efficiency. We studied power consumption in various commodity smartphones while playing YouTube videos with different mobility states in different cities. In this study, we observed that YouTube does not care about the instantaneous throughput or the radio state, which can cause a severe impact on power consumption.

With last observations, we design the EnDASH ABR algorithm that reduces power consumption during the video streaming by carefully scheduling the segment downloading based on the various device-related parameters, including cellular radio, battery, and device movement. EnDASH divides the entire playback session into slots. At the beginning of each slot, it first estimates the estimate the average throughput and then selects the maximum playback buffer length based on the slot based on the observations last slot. The buffer change in playback buffer length triggers the rescheduling in segment downloading. The EnDASH also has a module that decides the video quality for each segment, which further reduces power consumption. It requires three different learning engines, a) Throughput estimation engine, b) Buffer length prediction engine, and c) quality selection engine. The throughput estimation engine is based on random forest reinforcement learning. The buffer length prediction engine and quality selection engine are designed based on an actor-critic based recurrence neural network. We find that EnDASH provides better energy efficiency during evaluation than the existing ABR algorithm with a slight compromise in the QoE.

After saving energy, we decided to save some data usages. So, we developed FLiDASH, a federated adaptive live streaming system that exploits players' locality by forming coalitions. We observed that mega-event live streamers are localized, and most of the cases are connected via a local network. However, the current setup does not exploit the local network. Instead, every player downloads the segment on their own. FLiDASH utilizes the local network to find out players that can be communicated without involving the Internet. Then it forms a coalition among players that have similar quality targets. Every player in a coalition contributes to QoE's improvement by downloading based quality video it can download. We use the moving leader concept, where the leader is responsible for downloading a segment and deciding the leader for the next segment. These two algorithms work in a way so that a player with the lowest download speed gets to download the furthest segment in the long run. Our evaluation reveals that FLiDASH can improve the QoE for a coalition while reducing the server load and the network load. We also find that every player gains QoE and average quality by using FLiDASH than using any other ABR algorithms.

In the thesis, we analyze YouTube's adaptive streaming behavior and find interesting insights. We also analyzed the energy efficiency of YouTube and ABR's compatibility with different ABR algorithms. With study, we develop EnDASH to save energy in smartphones during video playback. Later we save data usages and server load by sharing segments among video players. In the future, we have to find an ABR algorithm for better compatibility with QUIC and TCP. It is also important to find a way to deploy a machine learning-based ABR algorithm, which is difficult now due to resource and technology requirements.

\textbf{Keywords:}
}

%%%%%%%%%%%%%%%%%%%%%%%%%%%%%%%%%%%%%%%%%%%%%%%%%
