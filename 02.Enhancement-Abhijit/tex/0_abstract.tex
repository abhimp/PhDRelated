
% As a general rule, do not put math, special symbols or citations
% in the abstract
\begin{abstract}
%With the ubiquity of devices to support Internet connectivity via multiple communication interfaces, like Wi-Fi and cellular connectivity over the smartphones, t
The nature of Internet traffic has changed dramatically within the last few years, where a large volume of traffic is originated from mobile applications (known as apps), web based multimedia streaming, computation offloading like cloud computing and Internet of Things (IoT) etc. These types applications generate multiple parallel short lived end-to-end connections. However, the three major requirements of todays' end-to-end traffic over the Internet, such as (a) support for mobility of devices, (b) capacity improvement through multi-path end-to-end transmissions, and (c) support for short-lived parallel connections, are not substantiated through the widely-deployed transmission control protocol (TCP). Further, the recent developments of multi-path TCP (MPTCP) as well as User Datagram Protocol (UDP) based Google's Quick UDP Internet Connections (QUIC) also fail to support all the above three requirements. MPQUIC, an extension of QUIC, is not pure multi-path because of the limitation of underlying UDP interface. As a consequence, in this work, we develop a new end-to-end transmission protocol, called \textit{Viscous}, to support the above three requirements over the Internet. Viscous is developed as a wrapper between the application and the transport layer, that works on top of the UDP and supports end-to-end reliability as well as congestion control while transmitting short-lived flows over multiple end-to-end paths. We introduce a number of novel concepts in Viscous, such as parallel and sequential flow multiplexing, decoupling of flow and congestion control etc. to overcome the problems associated with the current transport protocols. We have proposed a algorithm and its proof for priority based flow scheduling. It allows us to separate out flow and congestion control in different layers while maintaining the consistency between these two. Viscous has been implemented and tested over a variety of environments, and we observe that it can significantly boost up the performance of the end-to-end data transmission compared to TCP, MPTCP, QUIC and MP-QUIC.
\end{abstract}

%\begin{IEEEkeywords}
%	end-to-end transmission; TCP; MPTCP; QUIC; mobility; short flows	
%\end{IEEEkeywords}