\section{Introduction}

The Internet follows the conceptual design from a set of protocol suites where the major components are network layer protocol, termed as Internet Protocol (IP), and the connection based reliable transport layer protocol, known as Transmission Control Protocol (TCP), which popularly form the TCP/IP protocol suite that is the base for today's Internet architecture. However, during the last decade, it was well felt that the basic Internet architecture based on TCP/IP protocol suite is not suitable for growing demands of extending the Internet over a large number of smart devices at the edge, that spans from the smart handheld devices to the network of low power sensors and actuators, known as Internet of Things (IoT). The increasing number of edge devices reduces the scalability of the present Internet architecture, fundamentally because the nature of data traffic is different for today's Internet from the one that was envisioned during the development of the TCP/IP protocol suites~\cite{rexford2010future,zhang2016smart}. 

\subsection{Limitations of Current End-to-end Transport Protocols}
While TCP provides a reliable way for providing end to end connectivity, and because of its sender-controlled architecture, it gained a type of monopoly over the Internet, it has three fundamental problems, which can be summarized as follows.  
\begin{enumerate}
	\item  \textbf{Mobility}: Supporting seamless communication during mobility is one of the fundamental requirements for today's Internet. However TCP is a connection oriented protocol, a connection in TCP is identified by the tuple (source IP, destination IP, source port, destination port). If one of the source or destination IP changes, the ongoing TCP connection needs to be dropped a new connection needs to be established. TCP connection establishment and termination requires three handshake messages each, and therefore the signaling overhead for TCP is considerably high for a mobile device~\cite{Yadav2016}. 
	\item \textbf{Multihoming}: If a device is connected to Internet via multiple network interfaces, it is called a multihomed device. Multihome devices have the power to transmit or receive data via multiple paths through multiple physical interfaces. However, normal TCP can not utilize multiple interfaces available to a multihomes device~\cite{abdrabou2016experimental,de2016observing}. 
	\item \textbf{Short-lived Flows}: TCP employs a slow start phase to handle network congestion. However, a short-lived flow may fail to come out of the TCP slow start phase, resulting in underutilization of available network resources~\cite{de2016throughput,islam2016start}. Interestingly, many of the today's application over Internet, like web browsing, IoT communication, many smartphone applications (known as Apps) generate short lived flows. 
\end{enumerate}

Unfortunately, to the best of our knowledge, none of the today's transport protocols including TCP are able to handle all the three requirements as stated above. This motivates us to take this as a research problem which is challenging because of the huge dynamics in traffic loads over today's Internet. The types of traffic generated from multimedia streaming protocol is fundamentally different from the traffic generated from a Internet of Things (IoT) based event driven sensor and actuator network. Further, cloud computing brings a new dynamics, where the computation tasks are offloaded to the cloud servers connected with the Internet, and such applications generate huge number of parallel flows between the server and the client. This flows are short-lived, but the number of parallel flows as well as traffic generated over such flows is significantly huge, and the traffic generation pattern is also different from the normal Internet flows like web browsing. As mentioned, TCP as well as its variants are not suitable to handle such traffic dynamics over the Internet. Considering this, we look towards a new approach that essentially aggregates all the requirements and provides a modular design for and end-to-end protocol that is also programmable and dynamically tunable based on application needs. 

\subsection{Scope of Research Works}
A number of recent research works, such as~\cite{Yadav2016,abdrabou2016experimental,de2016observing,de2016throughput,islam2016start,maity2017tcp,liu2016improving} and the references therein, have revisited the TCP design fundamentals considering the needs for optimization at the transport layer protocol, so that the available network capacity can be fully utilized for both the event driven short-lived traffic as well as real time multimedia streaming traffic over the Internet. To support mobility at the transport layer, the naive approach is to decouple transport layer protocol from the IP. For this, the first approach is to solve it from the IP layer itself. So, the protocols like Mobile IP, \acrfull{hip}, \acrfull{shim6} have been developed. This kind of protocols hide the network changes from the transport layer. However, TCP congestion control algorithm is path dependent, and therefore, a sole IP layer based approach fails to provide the correct congestion feedback to the transport layer when the underlying path changes.  Consequently, the network community has explored end-to-end protocols to support the above mentioned features at the transport layer. To support mobility at the transport layer, a middleware between the transport and the IP layers has been developed, which helps to maintain the connection even if the device changes its network. TCP-Migrate~\cite{TCP-Migrate},  End to End Connection Control Protocol (EECP)~\cite{ECCP} supports this approach.  However, the above approaches can not solve the resource underutilization problem with multi-homing devices. {\em Multi-path TCP}~\cite{scharf2013multipath} has been developed for this purpose, where the connection between a sender and a receiver is established via multiple paths through multiple interfaces. A large number of recent works, such as~\cite{oh2016feedback,barik2016lisa,OLIARamin2012} and the references therein, have explored various aspects of MPTCP and measured its performance over dynamic Internet traffic scenarios. However, as explored in~\cite{kheirkhah2016mmptcp,kheirkhah2015short}, MPTCP does not perform well for short flows. 

To address the issue of short-lived flows, Dukkipati \textit{et. al.}~\cite{google-long-initcwnd} have suggested to use an initial congestion window size of at least $10$, so that the flows can come out of the slow start phase and become able to utilize the available bandwidth of the network. In another works~\cite{google-fast-open}, the authors have further suggested to utilize the TCP control packets for transmitting data, so that signaling overhead for connection establishment for short flows can be reduced. Later Google has developed an application layer protocol called SPDY~\cite{spdy}, that can multiplex multiple web requests over a single TCP connection. However, it suffers from the {\em Head of Line} (HOL) blocking issue, where if one or more packets drop or get lost, all the flows need to wait until TCP recovers the lost packets. To address the HOL blocking issue, Google has developed an User Datagram Protocol (UDP) based experimental protocol called {\em Quick Internet UDP Connection} (QUIC)~\cite{carlucci2015http,cui2017innovating}. QUIC is similar to SPDY, but it uses UDP as the transport layer protocol instead of TCP. QUIC can handle reliability, congestion control and flows control over the Internet as well as supports mobility. However, it does not have any control over the path it selects; and the path selection mechanism is completely dependent on the underlying routing algorithm. Therefore, QUIC is not a truly multipath protocol and does not support multihoming.


\subsection{Objectives and Current Direction of the Research Work}
In a nutshell, we observe that although the network community has understood that there is a requirement towards the development of a new end-to-end protocol that can address the ubiquity of devices as well as the heterogeneity of network traffic, the existing solutions fail to address those issues. As a consequence, in this research work, we plan to develop a multipath end-to-end protocol that can address the three fundamental shortcomings of the existing transport layer protocols -- mobility, multi-homing and short-lived event driven flows. We target to develop the protocol as a user level flow management protocol that runs on top of the UDP protocol, similar to Google's QUIC, however should be bundled with a number of new features. The reason for developing the use space architecture is to support application programmability so that the protocol modules can be tuned dynamically based on application requirement. As the first step towards the development of this end-to-end protocol over the Internet, we design and implement a base prototype, called \textit{Viscous}, that works as a middleware between the users' applications and the transport layer, and multiplex flows from multiple applications so that a single flow does not suffer from network bandwidth underutilization problem. Further, we design flow multiplexing in Viscous in such a way so that it becomes free from HOL blocking, and supports reliability as well as congestion control over unreliable UDP based end-to-end data transmission. We implement Viscous as a Linux application library, and test its performance over an emulated platform as well as a IoT networking testbed. We observe that Viscous can significantly improve the transport layer protocol performance compared to standard TCP, MPTCP and QUIC, when large number of short lived flows are generated from the edge devices. Our next target towards this direction is to extend Viscous with application layer programmability support and test its performance over a real IoT networking testbed. 

\subsection{Organization of the Report}
The rest of the report is organized as follows. In the next section, we discuss the details of various transport layer protocols and their limitations for supporting ubiquity over the current Internet. We also point out the current solutions explored in this direction along with their shortcomings. In Section 3, we provide the experimental evaluation of MPTCP for short lived flows to figure out the internals that can be helpful for the design of the new end-to-end protocol. Section 4 and Section 5 gives the detailed design and implementation details, respectively, for the newly developed Viscous protocol. We provide a thorough performance evaluation of Viscous over the \texttt{Mininet} architecture, as reported in Section 6. Finally, Section 7 concludes the report with a summary of next directions in this research work. 

%\newpage 
