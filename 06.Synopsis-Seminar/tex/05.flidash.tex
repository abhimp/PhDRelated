\section{Federated Adaptive Bitrate Live Streaming over Locality Sensitive Playback Coalitions}
\red{Live streaming is a unique service in the Internet which is alternative to TV broadcast yet more flexible. Few live streaming provider allow ability add custom delay. However, most popular flavor of live streaming is without any user defined delays. The live stream have a unique property of synchronous playback by millions of players. In case of mega event, we find another interesting property that many player belongs same local networks. There is two special case exists a) Internet on cable where users have different connection even though they are connected via local networks, b) shared Internet connection where multiple user/player share same Internet backbone. We want to develop a DASH based live streaming system where players forms coalition if they are connected via local network and stream the live video as a group. Incase of large network, Application-Layer Traffic Optimization (ALTO) server will be use to find appropriate players. Our goal is to improve quality for entire coalition and every player have to contribute towards the common goal.}
\subsection{Architecture}
Unlike DASH streaming system, FliDASH system have three components, these are a) streaming server: host the the video data, b) streaming client: plays the video, c) proximity server: identify nearby player. While the streaming server is unmodified webserver, the client is designed specially for live video streaming. It act as normal DASH based player if there is no other players nearby. However, if continuously look for nearby player to form coalition. As broadcast is confine to broadcast domain only, we use proximity server to find near by players. Before quality formation, clients first whether they compatible or not primarily by expected quality level. Once coalition is formed, coalition based video streaming system starts.
\subsection{title}