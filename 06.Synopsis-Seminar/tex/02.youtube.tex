\section{Adaptive Bitrate Streaming over Today’s Internet: A Case Study of YouTube}
To explore the interplay among various parameters that impact the ABR decision, we analyse YouTube vider streaming system. Although YouTube follows DASH guidelines, it deviates a lot from the it. For example it does not use stadard MPD file, instead use a self defined format. Similary there are several optimizations and parameters are not known. To understand those parameters we collect traffic trace of sizable amount of YouTube video via a controlled environment and try to indentify the unknown optimizations and parameters.
\subsection{Experimental setup}
In our experiment, we target we target web based YouTube player as the data collection platform. Our goal it collect the HTTP Archive (HAR) of as video session which is available in debug tool of browsers and {\tt tcpdump} trace. We also wants to control the network speed while playing video to understand the quality adjustment triggering point. We automate this entire setup using 3 tools, a) custom build throttler using the Linux library {\tt NetFilterQueue}, c) Firefox plugin {\tt har\_export\_trigger} (version 0.5.0-beta) to dump the HAR automatically without user invocation, and b) {\tt Selenium} to automate the video playback and trigger HAR export. Using out throttle we change the bandwidth from 200kbps to 2400kbps with a step of 200kbps.
\subsection{Observations}
\subsubsection{Parameter identification}
From the collected traces, we find that YouTube forward several parameters to the server with each request to fetch the video data. Among these parameters we find {\tt rbuf} and {\tt range} particularly interesting. {\tt rbuf} is remaining buffer for a particular {\tt itag} and {\tt range} is data segment (in bytes) from the original video. From {\tt rang}  we identify the video segment leghth (in seconds) requested in a HTTP request. We utilize these observation for further analysis.
\subsubsection{Insights into YouTube’s Bitrate Adaptation Algorithm}
With the help of identified parameters, we understand the the adaptation strategy upto certain degree. We find that YouTube take opportunistic approach downloading higher quality video when link quality goes up and take conservative approach to downgrade video quality while network quality drops. We also find that YouTube adapt segment length as per the network quality. When it finds steady network condition, it simply increase the segment length and decreases the same when it finds poor network condition.
\subsubsection{Data wastage}
Segment length adaptation may have far-reaching implications in terms of advantages for YouTube streaming, one of which is minimal data wastage. Since segment length increases gradually from a low value to higher values when bandwidth improves, overlaps between the segments of a lower quality and the next higher quality are largely diminished. This implies that data wastage values come down drastically (as opposed to a scenario with no segment length adaptation) – in our experiments, we compute the average wastage ratio, defined as $\frac{data\_downloaded - data\_played}{data\_played}$, to be $0.82\times10^{-6}$. This is in sharp contrast with previously reported values.
