\section{Introduction}
The Internet, which connects systems all over the globe, has become part of the modern lifestyle. The basic services like telephony to complicated procedures like telesurgery nowadays require the Internet. While it provides various services, video streaming is the most prominent one over today's Internet. According to Cisco Visual Networking Index (VNI) predictions\footnote{\url{https://www.cisco.com/c/dam/m/en_us/network-intelligence/service-provider/digital-transformation/knowledge-network-webinars/pdfs/1213-business-services-ckn.pdf} (Access: \today)}, by 2022, almost half of the Internet-connected devices are going to be video capable, which will contribute 82\% of all the Internet traffic. Also, the high definition video demand will increase by many folds, as more connected devices will be 4K capable. To satisfy this massive growth in video streaming demand, every parts of the network, starting from the hardware to the software, need to be specially optimized for supporting massive-scale online video streaming. There are three different types of online video streaming systems which are viable over the Internet -- (a) \textit{interactive video streaming}, (b) \textit{live video streaming}, and (c) \textit{static} or \textit{video-on-demand} (VoD). Among these three categories, live, and VoD streaming contributes to significant traffic on the Internet, as they mostly demand high-definition videos. While services like YouTube, Twitch, etc. support both these types of video streaming, NetFlix, Amazon Prime Videos, etc. are primarily concentrated on supporting the VoD service. 

In earlier days, online video streaming used to utilize more dedicated protocols like \textit{Real Time Protocol} (RTP), \textit{Secure Real Time Protocol} (SRTP), \textit{Real Time Streaming Protocol} (RTSP), etc. These protocols are push-based, which means the servers push the video content towards the clients, and a client almost has no control over it. Due to being pushed-based protocol, most of the network middleboxes like proxies and network address translation (NAT) usually block them (intentionally and unintentionally). To overcome these issues, YouTube started delivering video over Hypertext Transfer Protocol (HTTP) in 2005. As the HTTP is used to serve the World Wide Web (WWW), it was widely accepted by NAT boxes and proxies, and video data can pass through without any problem. Soon, video streaming over HTTP became a defacto standard for the web-based video player. Initially, the HTML did not support any video playback mechanism, so Adobe flash player was used as it supported the progressive video file download and playback of partially downloaded video files~\cite{gill2007youtube}. HTML embedded audio-video playback capabilities were introduced in HTML5, and all the web-based online video streaming services moved towards the HTML5 video playback.

In 2009, Apple released HTTP Live Streaming service for their QuickTime video player. It included a feature, adaptive bitrate (ABR), where video quality can be adapted based on the observed network quality to avoid rebuffering. Soon almost all the services started service adaptive video streaming. The different organizations have developed different techniques for adaptive video streaming over HTTP. Among them, \textit{Dynamic Adaptive Streaming over HTTP} (DASH) by MPEG~\cite{ISO/IEC23009-1:2019} is notable as it has an open-source implementation by DASH Industry Forum (DASH-IF). HTML5 has added an experimental extension called Media Source Extension (MSE)\footnote{\url{https://www.w3.org/TR/media-source/} (Access: \today)}, supporting the adaptive video. 

%While video streaming is very popular over the web, it is also popular with smartphone users. Most smartphone operating systems, including Android, IOS, and Windows, support adaptive video streaming in both HTML5 based video players through the browser and native video player for the application. Furthermore, most of the online video services started providing an application to run on such smartphones as well as the support for their website in the mobile version.

Currently, most the the streaming providers support adaptive video streaming. Any adaptive streaming system runs an adaptive bitrate (ABR) algorithm to adapt video quality based on the network quality. ABR algorithms are the most critical part of a video streaming system as the \textit{quality of experience} (QoE) depends on them. Video QoE is a combination of three parameters~\cite{yin2015control}, -- (i) the overall quality, (ii) quality fluctuations or lack of smoothness, and (iii) rebuffering. Any ABR is supposed to improve QoE by maximizing the overall quality and minimizing the quality fluctuations and the rebuffering. At the inception of ABR video streaming, the ABR algorithm was simple; it used to match the video bitrate to the observed throughput~\cite{5677508,10.1145/1943552.1943575,10.1145/1943552.1943574}. However, this technique causes frequent fluctuations in the bitrate, which can cause irritation to the users. So, the researchers started improving the ABR algorithms with various techniques. Over the decade, the throughput-based algorithms were improved by the buffer-based~\cite{Spiteri2016,10.1145/2910017.2910596,7393865} and later hybrid -- a combination of throughput and buffer-based algorithms~\cite{7247436,140405,yin2015control,10.1145/2670518.2673877}. Very recently, to improve the QoE even more, machine learning (ML)-based\cite{Mao2017,Akhtar2018,9155492} ABR algorithms have been proposed. However, as the number of streaming media users grows day by day, and with having numerous video streaming apps available over smartphones, new challenges are being observed, demanding several optimizations in the current ABR streaming systems. 

%The online video streaming system is very popular, and adaptive video streaming over HTTP is its core. The adaptive video streaming is network friendly as it can reduce its bandwidth requirement by choosing a lower bitrate allowing other applications to go through a shared bottleneck. When multiple players share a common bottleneck link, they automatically adjust their video quality to allow other players to stream with reasonable quality. Over time, all the player gets an almost fair share of the bottleneck link.

%======================
\subsection{Motivation of the Research}
The motivation behind this research comes from different facts, as we summarize below. 

\subsubsection{ABR Algorithms and Their Interplay with Computer Networks}
ABR algorithms are an essential part of any adaptive video streaming system, as discussed above. It is assumed that ABR algorithms treat the streaming and other types of Internet traffic fairly in most cases. However, during the COVID-19 pandemic, various video streaming services started dropping the highest quality of video from their service to preserve the bandwidth available to the Internet Service Providers (ISPs), as many people began working from home. At that time, the research community started questioning whether it was a fault of the video streaming system, or whether or not the existing ABR algorithms adjust themselves to the lower bitrates when the load is high or something wrong with the deployment of those streaming services. While we do not know the answer, it is clear that the ABR algorithms are fundamental, and the video streaming systems need to adapt to various network conditions. Questions like these motivate us to research the current commercial video streaming systems to explore how fairly they work over today's Internet.

Google's YouTube is one of the pioneers of video streaming over HTTP(S) and a major online video streaming player. It serves a billion videos to 30 million users daily\footnote{\url{https://blog.youtube/press/} (Access: \today)}. To scale the service to support these huge user-bases, YouTube has developed several improvements that involve content delivery networks, transport protocol, and adaptation algorithms. There is no doubt that YouTube gained the attention of the research community to understand the system. In the past, several pieces of research have studied YouTube for a better understanding of its functionalities and its impact on Internet traffic. However, most of the work either explore the traffic patterns and QoE properties of YouTube~\cite{gill2007youtube,krishnappa2013dashing,wamser2016modeling,wamser2015poster,6757893ieeeexp,7129790ieeeexp} or explore the trade-off between video quality and data wastage~\cite{sieber2015cost,seufert2015youtube,sieber2016sacrificing}. Although these studies give an important aspect of YouTube, they failed to explore the internal parameter and streaming strategies. It is crucial to know the adaptation strategies and internal parameters of YouTube to improve the DASH-based system.

On the other side, the web being Google's primary business, it started a project called \textit{`Make the Web Faster'}\footnote{\url{https://developers.google.com/speed} (Access: \today)}, where it developed several tools and optimization to load a web page faster. \textit{Quick UDP Internet Protocol} (QUIC)\cite{langley2017quic} is a product of this project where Google ditched the underlying Transmission Control Protocol (TCP) and provided a new transport protocol based on the User Datagram Protocol (UDP). Starting from the release of QUIC, researchers have studied the performance of it in various scenarios, including DASH-based video streaming. As QUIC works differently than TCP, and most of the ABR algorithms used in different video streaming platforms are designed for TCP as the transport protocol, those ABR algorithms might not work well with QUIC. Research works like \cite{bhat2018improving,van2018empirical} studied the impact of QUIC over various ABR algorithm and found that buffer-based ABR algorithms are not well suited for QUIC. However, those works are performed using old ABR algorithms, and the studies are mostly simulation-based. Also, these works do not provide any insight into the root cause of the differences.

\subsubsection{Energy Efficiency in Online Video Streaming}
As per YouTube's report, users love to play online videos on a smartphone, which is also true for other streaming platforms. However, video streaming is inherently an energy-consuming service. The uneven distribution of Long Term Evolution (LTE) cellular network base stations, clustered crowds, and non-optimized video app can make these things even worse. Research in \cite{10.1145/2910018.2910656} shows plenty of places to save energy by optimally configuring the streaming parameters like segment length, maximum buffer length, and segment download scheduling. Such research indicates that the interplay between the cellular network and the device's radio controller plays a significant role in the device's energy efficiency during online adaptive video streaming. It is important to learn various cellular network parameters and make informed adaptive decisions based on the learned parameters to make the adaptation more energy efficient.

\subsubsection{Live Streaming and Its Impact}
Live streaming is gradually becoming an alternative to televisions (TV) as most mega-events' live videos are broadcasted over the Internet using online live streaming. As more users shift towards online live streaming, it becomes difficult to scale such mega-events' streaming. The existing literature proposes solutions using Peer-to-Peer (P2P) networks to scale the live video streaming as players are in sync. However, P2P based live streaming suffers as most of the ISPs allocate significantly less uplink bandwidth than downlink bandwidth to their subscribers. 

When a P2P network spans over the Internet, traffic needs to go through different ISPs and autonomous systems (ASes), and this traffic will count towards the user's data connection. So, users may not take it lightly. However, in mega-events, many users stay in clusters and share a common Internet backbone. An intelligently developed service can try to exploit the internal network to extend the service and provide better quality with lower overhead to the streaming server and the ISP. This gives plenty of scopes to optimize the live streaming systems for ABR-based streaming.


%======================
\subsection{Objectives of the Thesis}
The objectives of this thesis are as follows. 

\subsubsection{Understanding Adaptive Streaming Services over Today's Internet}
As we mentioned, almost all the over-the-top (OTT) streaming media services over the Internet has adopted DASH and use some form of ABR algorithms at the client application. Given the context and importance of video streaming services, it is necessary to understand how they work and how they interact with the web and other traffic over the Internet. It is also essential to understand the interactions between the underlying networking protocols, like QUIC, and the ABR streaming services running as Internet applications. With these requirements in mind, this thesis's first objective is to analyze YouTube streaming media services to explore the ABR algorithms and how it interacts with the various networking protocols. Although the adaptation mechanism in YouTube is based on DASH, we want to find the differences between the conventional DASH principles and the streaming algorithm used by YouTube. YouTube is a closed source commercial OTT application and uses proprietary services; the only way to understand the ABR mechanism of YouTube is by reverse-engineering the system based on observations of its behavior. 

\subsubsection{Explore the Impact of Network Protocols and Client Mobility on the Performance of ABR Techniques}
The ABR algorithms' performance depends on the observed network quality; however, it highly depends on the underlying transport layer, TCP, until Google developed QUIC. Our next objective is to explore the impact of the underlying transport protocols, such as TCP and QUIC, on the behavior of various modern ABR techniques~\cite{Spiteri2016,Mao2017}. We further envision exploring ABR streaming algorithms' energy-efficiency, as a majority of the popular OTT media services run as a client application over smartphones, and users prefer to watch videos while in transit. Consequently, the impact of mobility on smartphone energy-efficiency due to the ABR algorithms' network traffic generation patterns is an essential field of study. Our objective is to explore the same over diverse networking environments.  


\subsubsection{Energy Efficient Video Streaming over Smartphones}
While QoE is the primary goal for any ABR algorithms, as mentioned before, looking into secondary objectives such as energy efficiency during video streaming is also essential when the services run over devices like a smartphone. Power management in a smartphone is crucial, which is also important from the perspective of video streaming, especially when users are in mobility. So, we aim to develop an ABR algorithm, especially for smartphone-based video streaming, to reduce battery consumption while maintaining reasonable QoE. We aim to utilize the radio-related information available in a smartphone, apart from various other information such as GPS location, device speed, battery information, etc. This primary objective can be subdivided into multiple goals as follows.
\begin{enumerate}
	\item To find the correlation between radio, battery, and other physical parameters of a device with the network quality.
	\item To utilize these parameters to develop an ABR algorithm to reduce energy consumption while video streaming.
	\item To provide reasonable QoE based on the device setting, although it might sacrifice the finest QoE to preserve energy.
\end{enumerate}

\subsubsection{Efficient ABR for Live Streaming}
This work aims to reduce Internet and server load while maintaining high QoE during live streaming by exploiting the feature of synchronous playback and locality information within ABR. Our objective is to design a live streaming player that can find and share video segments with other players in the locality by dynamically forming a peer-to-peer (P2P) network during the online video streaming. The challenge here is to maintain the synchronous playback among the players through ABR streaming, as different players download different video segments and share them over the P2P network. To support ABR over such an architecture, the players need to decide the playback bitrate for the next video segment collectively and also have to schedule the segment downloads among the players by maintaining fairness in the P2P network. QoE optimization needs to be the overall goal for this collective ABR decision; however, each player needs to coordinate with others to achieve the same. 

%======================
\subsection{Contributions of the Thesis}
This thesis consists of four contributory chapters. The significant contributions of the individual chapters are given below. 

\subsubsection{Adaptive Bitrate Streaming over Today's Internet: A Case Study of YouTube}
In this chapter, we perform reverse engineering over the YouTube ABR streaming algorithm to find out how it works and what are the parameters that impact the performance of ABR streaming over a large-scale Internet scenario. The contributions in this chapter are as follows. 
\begin{itemize}
	\item We illustrate a methodology to study YouTube's adaptive streaming behavior in-depth -- we identify and closely study the interplay among important parameters enabling this streaming algorithm.
	\item Our experiments reveal that YouTube adapts the {\it segment length} parameter before attempting to adjust video resolution -- a phenomenon not reported in the literature.
	\item We observe that segment length adaptation leads to much lower data wastage values on average than the ones reported by prior studies.
	\item We propose an analytical model, augmented with a machine learning-based classifier, which enables prediction of data consumption for an initial playback video quality when it is possible to estimate the network conditions a priori using existing mechanism like~\cite{Zou2015}.
\end{itemize}

\subsubsection{Transport Protocols and Mobility Choices for ABR Streaming: Performance vs. Energy Efficiency}
In the second contributory chapter of the thesis, we analyze various state-of-the-art ABR techniques under different transport protocols and mobility choices. This analysis gives us an idea about how ABR algorithms work in diverse scenarios, evident over today's Internet. The contributions in this chapter are summarized below. 
\begin{itemize}
	\item Like the previous chapter, we first conduct a study over YouTube to analyze YouTube ABR performance over two different transport protocols -- TCP and QUIC.
	\item We then perform a study over a controlled environment, to explore how different modern ABR algorithms like BOLA~\cite{Spiteri2016}, MPC~\cite{yin2015control}, Pensieve~\cite{mao2017neural}, etc. work over QUIC in comparison to TCP.
	\item Finally, to understand the impact of ABR streaming over the energy consumption of a mobile device, we conduct a thorough experiment using commodity smartphones and commercial cellular network connections under different mobility scenarios. We observed that arbitrary data download through ABR may impact the smartphone's energy consumption behavior when some OTT media application runs for streaming a video. 
\end{itemize}

\subsubsection{EnDASH - A Mobility Adapted Energy Efficient ABR Video Streaming for Cellular Networks}
This chapter develops an energy-efficient video streaming algorithm while preserving the QoE performance of the ABR mechanism. The salient contributions of this chapter are as follows. 
\begin{itemize}
	\item With the observation from previous chapters, we developed a throughput prediction engine to predict average throughput for a short future.
	\item We design an ABR algorithm based on the throughput prediction algorithms and instantaneous parameters to reduce the power consumption by optimally adjust the playback buffer lengths.
	\item We performed thorough experiments in an emulated environment to show the performance of the proposed mechanism, called \textit{EnDASH}. We observe that the proposed mechanism significantly saves energy consumption during video streaming with marginal compromise in the overall QoE. 
\end{itemize}

\subsubsection{Federated Adaptive Bitrate Live Streaming over Locality Sensitive Playback Coalitions}
In the final contributory chapter of the thesis, we focus on ABR for the live streaming of videos and develop an architecture to support efficient adaptive live streaming over the Internet. The contributions of this chapter are as follows. 
\begin{itemize}
	\item In this chapter, we have identified the possibility of exploiting the local network to reduce server load during live streaming, which improves the QoE for the streaming application.
	\item The work proposes a novel way to share segments among users reasonably and efficiently by forming a coalition among nearby and similar types of video players.
	\item It also proposes a novel algorithm to select bitrates adaptively to increase the QoE of the entire coalition simultaneously, instead of individual QoE only.
	\item We have evaluated the proposed system over an emulation environment, and we observe that our system can significantly improve the live streaming QoE compared to other baselines. 
\end{itemize}

\subsection{Organization}

Henceforth, this synopsis report of the thesis is organized as follows. Section 2 gives a broad overview of the existing literature and its limitations. Section 3 highlights our contributions in exploring the YouTube ABR mechanism over today's Internet and the importance of various parameters that govern the YouTube streaming algorithm. In Section 4, we summarize our observations in exploring the protocol and mobility choices for designing ABR algorithms while keeping energy-efficiency as an objective. Section  5 discusses EnDASH, and adaptive streaming mechanism, which considers energy-efficiency as a goal while developing the ABR. Section 6 constitutes the summary of FLiDASH -- an adaptive live streaming mechanism to improve the QoE while reducing the network overhead. Finally, Section 7 gives the thesis's organization, and Section 8 concludes the report by highlighting the publications out of the thesis in Section 9.  

\section{Literature Survey}
To enhanced the online streaming experience, researchers have developed various systems and ABR algorithms. We categorize the ABR algorithms broadly in -- (a) classical ABR algorithms, and (b) learning based ABR algorithms. In the following sections, we discuss both of these ABR algorithms. Finally, we discuss the ABR algorithms specially designed for live streaming and smartphones.

Several classical ABR algorithms have beed developed over the years. It starts with simple throughput driven ABR algorithms, where ABR directly matches the streaming quality to the last observed network throughput~\cite{5677508,10.1145/1943552.1943575,10.1145/1943552.1943574}. Although it reduces the rebuffering, quality change is imminent. So researchers tried to tune the throughput measurement and quality switching policy to improve the QoE. However, throughput-based algorithms require continuous throughput measurement, which is impossible due to the ON-OFF traffic pattern. The fine-grain throughput measurement is not possible from the application layer due to the network stack's nature. So, researchers started developing buffer-based ABR algorithm as the buffer can be an indication of throughput. Buffer-based algorithms adapt the bitrate with or without having a bitrate map. However, the frequent quality switch is still a problem. So both throughput-based and buffer-based algorithms applied mechanisms to avoid strict changes in the quality. While throughput and buffer-based ABR algorithms provide reasonably better QoE, there are places to improve. It cannot always provide the best solution due to the nature of the network dynamics -- a Hybrid solution can rescue ABRs from those pitfalls. ABR algorithms in \cite{7247436,140405,yin2015control,10.1145/2670518.2673877} perform much better than any buffer-based or throughput-based algorithms. These algorithms assume the QoE maximization as an optimization problem and try to solve it. However, there are problems in deploying hybrid ABR algorithms as these are complex, computationally heavy, and require a special solver to solve the problem. Although algorithm designers usually provide a lightweight solution, those are not adequate.

Machine learning (ML) and deep learning (DL) can perform better than classical systems if trained correctly. The ABR algorithms, like Pensieve~\cite{Mao2017}, Oboe~\cite{Akhtar2018}, etc., show the potential of the machine learning algorithms. The idea of super-resolution~\cite{9155384} and liveClip~\cite{10.1145/3386290.3396937} gives another perspective to the video streaming paradigm. While all these ABR algorithms work well in the evaluation, these are not easy to deploy widely. Most of the ABR algorithm needs to train for a long time before they can be used. The training time is enormous and computationally heavy. Training needs to be done frequently and sometimes, per video. Another big hurdle in the deployment of these algorithms is their library dependency. To infer the correct decision from a pre-trained model, an ML algorithm needs to run in the player. Library supports are essential to run ML algorithms in players, and not all the different platforms support all the libraries. Inference can be made by offloading computation to a server. However, it is incredibly inefficient as it involves network delay. We feel there is much work needed before any of the algorithms can be deployed widely and efficiently.

Apart from the design of ABR algorithms, a few works in the literature have also explored ABR performance of existing OTT media services like YouTube. Existing studies on YouTube video streaming system and video QoE can be grouped into two broad classes. The first class of works explore traffic patterns and video QoE properties of YouTube~\cite{gill2007youtube,krishnappa2013dashing,wamser2016modeling,wamser2015poster,6757893ieeeexp,7129790ieeeexp}. These works mostly study YouTube's behavior at the periphery, which, although provides a summary of performance metrics, fails to say much about the internals of YouTube's video streaming protocol. The second class of studies, however, explore the adaptive streaming characteristics of YouTube. In \cite{finamore2011youtube}, the authors investigated YouTube's data delivery system from the end-user view. They illustrated evidence of massive wastage of downloaded data since viewers often do not watch entire videos -- the study, however, was performed at a time when YouTube used progressive download as the streaming mechanism and is therefore stale. \cite{krishnappa2013dashing} is probably the first work to evaluate YouTube's performance since its adoption of adaptive streaming -- the authors claim that YouTube gains $83\%$-$95\%$ in terms of bandwidth by switching from progressive download to DASH. Some recent works~\cite{sieber2015cost,seufert2015youtube,sieber2016sacrificing} have studied YouTube's DASH behavior to analyze the trade-off between quality and data wastage -- however, their approximations lead to gross overestimation. They perform controlled experiments by varying the underlying link bandwidth and compute wastage.

