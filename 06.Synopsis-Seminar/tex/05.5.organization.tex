%======================
\section{Organization of the Thesis}

In this section, we provide a brief description of the organization of the thesis.
\begin{itemize}
	\item {\bf Chapter 2} provides the details of online video streaming system and different ABR algorithms developed in the past. Moreover, it iterates over the merits and demerits of the proposed solution to the problem broadly linked with this thesis.
	\item {\bf Chapter 3} presents the study on YouTube video streaming service and analyses the adaptation technique taken by YouTube.
	\item {\bf Chapter 3} performs a study to analyze the performance difference between two transport protocols, TCP and QUIC, on the adaptive bitrate video streaming system, with both YouTube and standard DASH player from DASH Industry Forum. We further analyze the energy consumption by a smartphone during online video streaming in various mobility scenarios.
	\item {\bf Chapter 4} presents a novel ABR algorithm, called EnDASH, to reduce the energy consumption during the online streaming by optimizing the segmented schedule in buffer length.
	\item {\bf Chapter 5} proposes a DASH-based live video streaming system which exploits the locality information to improve the video QoE and decreases the network usage by sharing segment among local players. It also describes a novel approach to form a coalition among the local players and decides the bitrate for the entire coalition.
	\item {\bf Chapter 7} concludes the thesis by summarizing our contributions and listing down the possible future directions.
\end{itemize}